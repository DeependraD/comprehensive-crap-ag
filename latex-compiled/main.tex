\documentclass[a4paper,12pt]{exam}
\usepackage[top=1in, bottom=1in, left=0.75in, right=0.75in]{geometry}
%tlmgr install mathpazo
%tlmgr install setspace
%tlmgr install collection-fontsrecommended
\usepackage{mathpazo} % mathpazo fonts are dark and easily seeable
% \usepackage[T1]{fontenc}
% \usepackage[ansinew]{inputenc} % produces error when used with xelatex
\usepackage[singlespacing]{setspace}
\usepackage{graphicx}
\usepackage{amsmath,amssymb,amsfonts}
\usepackage[ddmmyyyy]{datetime}
\usepackage{environ}
\usepackage[normalem]{ulem}
\usepackage{etoolbox}

% % uncomment this to format exactly as question paper of exams
% \title{\vspace{-1.5cm}
%   {\huge \textbf{General Knowledge: QnA and Subjective}}\\[3mm]
%   {\LARGE Practice}\\[3mm]
%   {\LARGE All topics}\\[1mm]
%   \normalsize
%   \begin{tabular}{lll}
%       Name: ~\rule{3cm}{1pt} & Roll no.:~\rule{2cm}{1pt} & Date:~\rule{2cm}{1pt} \\[2mm]
%       Full Marks: yy & & Time: x hours \\
%   \end{tabular}
%   \hrule
% }

% comment 'title' when combining with main book as a part or avoid including first page
\title{\vspace{-1.5cm}{\huge \textbf{General Knowledge: Agriculture}}\\[2mm]
\hrule
}
\date{}

% % print correct answers and fillins
\printanswers
\CorrectChoiceEmphasis{\bfseries\boldmath}

% % print marks weight in the right
\pointsinrightmargin
\bracketedpoints

% adding gradetable
\addpoints

% the \llap command to correct alignment in a question with parts.
\renewcommand{\questionlabel}{\llap{Q}\,\thequestion.}
\unframedsolutions % put solution inside a box

\pagestyle{empty} % turn off page number, this is useful when binding the pdf to knitted document

%% following code is solution to aligning multiple choices in single line and
%% displaying answers in line after dotted marks
%% copied from: https://tex.stackexchange.com/a/141643

\newlength\answerspace
\setlength\answerspace{0.5in}
\newcommand\dottedanswerline[1][{}]{%
  % One optional argument, the default value of which is empty.
  \unskip\linebreak[0]\enspace
  \hbox{}\nobreak\dotfill
  \ifprintanswers
    \hbox to \answerspace{(\hfil#1\hfil)}%
  \else
    \hbox{(\hskip 0.5in)}%
  \fi
}% dottedanswerline

\makeatletter
\newlength\choiceitemwidth
\newif\ifshowsolution \showsolutiontrue
\newcounter{choiceitem}%

\def\thechoiceitem{\Alph{choiceitem}}%

\def\makechoicelabel#1{#1\uline{\thechoiceitem.}\else\thechoiceitem.\fi\space} %underline the answer item label if we want to print the answer

\def\choice@mesureitem#1{\cr\stepcounter{choiceitem}\makechoicelabel#1}%

%measure the choices, this is the first time we need to parse the \BODY
\def\choicemesureitem{\@ifstar
  {\choice@mesureitem\ifprintanswers \xappto\theanswer{\thechoiceitem}\ignorespaces}%
  {\choice@mesureitem\iffalse}}%

\def\choice@blockitem#1{%
  \ifnum\value{choiceitem}>0\hfill\fi
  \egroup\hskip0pt
  \hbox to \choiceitemwidth\bgroup\hss\refstepcounter{choiceitem}\makechoicelabel#1}

\def\choiceblockitem{\@ifstar
  {\choice@blockitem\ifprintanswers\ignorespaces}%
  {\choice@blockitem\iffalse}}

\def\choice@paraitem#1{%
  \par\refstepcounter{choiceitem}\makechoicelabel#1}

\def\choiceparaitem{\@ifstar
  {\choice@paraitem\ifprintanswers\ignorespaces}%
  {\choice@paraitem\iffalse}}


\NewEnviron{items}{%
  \def\theanswer{}
  \begingroup
    \let\item\choicemesureitem
    \setcounter{choiceitem}{0}%
    \settowidth{\global\choiceitemwidth}{\vbox{\halign{##\hfil\cr\BODY\crcr}}}%
  \endgroup
  % \dottedanswerline[\theanswer] % comment this line to display underline for in \printanswers mode, uncommenting this will also show correct answer printed on the side of question following a dotted line.
  \trivlist\item\relax
  \parindent0pt
  \setcounter{choiceitem}{0}%
  \ifdim\choiceitemwidth<0.25\columnwidth
    \choiceitemwidth=0.25\columnwidth
    \let\item\choiceblockitem
    \bgroup\BODY\hfill\egroup
  \else\ifdim\choiceitemwidth<0.5\columnwidth
    \choiceitemwidth=0.5\columnwidth
    \let\item\choiceblockitem
    \bgroup\BODY\hfill\egroup
  \else % \choiceitemwidth > 0.5\columnwidth
    \let\item\choiceparaitem
    \BODY
  \fi\fi
  \endtrivlist
}
\makeatother

\begin{document}

\maketitle

% \vspace{-3cm}

% % uncomment this to set instructions for answering the questions
% \section*{Instructions}
%
% \begin{enumerate}
%   \item All questions are compulsory
%   \item Calculator is not allowed
%   \item All questions carry equal marks.
%
% \end{enumerate}
% \hrule \vspace{5mm}

% \gradetable


\subsection*{\fullwidth{\Large \centering \textbf{General Knowledge Multiple Choice: Agriculture (Set 1)}}}

\begin{questions}

\question Which of the following agricultural produce is not identified as a potential export commodity for Nepal ?
  \begin{choices}
  \choice tea
  \choice organic honey
  \choice chayote
  \choice cardamom
  \end{choices}

\question Scattered diagram is used to see:
  \begin{choices}
  \choice Correlation
  \choice Mean
  \choice Variance
  \choice Range
  \end{choices}

\question Linkage between agro-industries and farmers is constrained by:
  \begin{choices}
  \choice Scattered production pocket
  \choice Small scale production
  \choice Lack of grading and standardization
  \choice All of above
  \end{choices}

\question In the year 2010/11, the percentage contribution of agriculture, forestry and fishery sectors on GDP was estimated at about:
  \begin{choices}
  \choice 28\%
  \choice 36\%
  \choice 46\%
  \choice 56\%
  \end{choices}

\question Zero tillage technology is mainly practiced in:
  \begin{choices}
  \choice Wheat
  \choice Rice
  \choice Maize
  \choice Lentil
  \end{choices}

\question Altering genetic make-up of plants by the low temperature is called:
  \begin{choices}
  \choice Vernalization
  \choice Freezing
  \choice Thawing
  \choice Hydrolisation
  \end{choices}

\question In-situ conservation refers to the conservation of germplasm under:
  \begin{choices}
  \choice Indoor lab condition
  \choice Gene bank
  \choice Natural conditions
  \choice Soil surface
  \end{choices}

\question \textit{Phalaris minor} weed is a major weed of:
  \begin{choices}
  \choice Maize
  \choice Wheat
  \choice Rice
  \choice Chickpea
  \end{choices}

\question Replication is essential to reduce:
  \begin{choices}
  \choice Degree of freedom
  \choice Coefficient of variation
  \choice Significance level
  \choice Experimental error
  \end{choices}

\question Among the several agricultural commodities which of the following crop is the most important in terms of nationally exported commodities?
  \begin{choices}
  \choice Ginger
  \choice Lentil
  \choice Sugarcane
  \choice Tea
  \end{choices}

\question Area under jute cultivation is highest in:
  \begin{choices}
  \choice Jhapa
  \choice Morang
  \choice Saptari
  \choice Kailali
  \end{choices}

\question Seed priming is done to:
  \begin{choices}
  \choice Dry seed
  \choice Seed wetting
  \choice Increase germinability
  \choice Sorting seeds
  \end{choices}

\question The type of layout that accomodates the highest number of fruit plants (saplings) is:
  \begin{choices}
  \choice Rectangular
  \choice Square
  \choice Hexagonal
  \choice Contour
  \end{choices}

\question Rooting stimulant plant growth regulator is:
  \begin{choices}
  \choice Gibberelin
  \choice Kinin
  \choice Ethylene
  \choice Auxin
  \end{choices}

\question The original habitat of Arabica coffee is:
  \begin{choices}
  \choice Argentina
  \choice Peru
  \choice Ethiopia
  \choice Brazil
  \end{choices}

\question The pineapple is propagated through \fillin[][3cm]
  \begin{choices}
  \choice Sexual method
  \choice Asexual method
  \choice Succers
  \choice All of above
  \end{choices}

\question The appropriate type of layout for establishing an orchard of fruit in a slopy land is:
  \begin{choices}
  \choice Rectangular
  \choice Square
  \choice Hexagonal
  \choice Contour
  \end{choices}

\question Apple stored in \fillin[][3cm] region is stored for longest.
  \begin{choices}
  \choice Mid hills
  \choice High hills
  \choice Terai
  \choice All of above
  \end{choices}

\question The example of ex-situ conservation of plant genetic resources is:
  \begin{choices}
  \choice Botanical garden
  \choice Field gene bank
  \choice Data bank
  \choice All of above
  \end{choices}

\question Panel on climate forecasted that the atmospheric temperatures will rise by 1.8-4.0 degree Celcius globally by:
  \begin{choices}
  \choice 2080 AD
  \choice 2090 AD
  \choice 2100 AD
  \choice 2110 AD
  \end{choices}

\question Which division (Mahasakha) is responsible for planning and implementing Agriculture, environment and agro-biodiversity related programs?
  \begin{choices}
  \choice Planning division
  \choice Monitoring and evaluation division
  \choice Gender equity and environment division
  \choice Agribusiness promotion and statistics division
  \end{choices}

\question Which type of erosion is the most hazardous ?
  \begin{choices}
  \choice Rill erosion
  \choice Gully erosion
  \choice Sheet erosion
  \choice All of above
  \end{choices}

\question When the prospective yields are discounted by the marginal efficiencies of capital, the product value is equal to:
  \begin{choices}
  \choice The demand price
  \choice Rate of profitability
  \choice Supply price of assets
  \choice Marginal productivity of capital
  \end{choices}

\question An increase in the general price level of an economy is called:
  \begin{choices}
  \choice Inflation
  \choice Deflation
  \choice Depression
  \choice None
  \end{choices}

\question Which of the following interprets perfect negative correlation between two variables ?
  \begin{choices}
  \choice $r = \pm 1$
  \choice $r = + 1$
  \choice $r = - 1$
  \choice $r = 0$
  \end{choices}

\question "One village one product" program has been conducted for fish in:
  \begin{choices}
  \choice Saptari
  \choice Dhanusha
  \choice Bara
  \choice Rupandehi
  \end{choices}

\question Which of the following is not a type of sprayer ?
  \begin{choices}
  \choice Hand compression sprayer
  \choice Power sprayer
  \choice Blow sprayer
  \choice Trigger sprayer
  \end{choices}

\question Production decrease due to weeds in different crops is:
  \begin{choices}
  \choice 10\%
  \choice 15\%
  \choice 20\%
  \choice 25\%
  \end{choices}

\question Broomrapes are:
  \begin{choices}
  \choice Root parasite
  \choice Shoot parasite
  \choice Fruit parasite
  \choice Leaf parasite
  \end{choices}

\question Silk production in developed countries has been slowly coming down, mainly due to:
  \begin{choices}
  \choice Decreased global demand
  \choice Increased labor cost
  \choice Increased diasease and pest threat
  \choice Climate change effects
  \end{choices}

\question Scab disease is mostly observed in:
  \begin{choices}
  \choice Apple
  \choice Mango
  \choice Litchi
  \choice Banana
  \end{choices}

\question Most honey producing honey bee is:
  \begin{choices}
  \choice \textit{Apis dorsata}
  \choice \textit{Apis florae}
  \choice \textit{Apis cerena}
  \choice \textit{Apis melifera}
  \end{choices}

\question Which one of the following is an entopathogenic fungi?
  \begin{choices}
  \choice \textit{Metarhizium anisopliae}
  \choice \textit{Beauveria bassiana}
  \choice \textit{Entomophthora} spp.
  \choice All of above
  \end{choices}

\question Metribuzin is one of the effective:
  \begin{choices}
  \choice Herbicides
  \choice Nematicides
  \choice Fungicides
  \choice Bactericides
  \end{choices}

\question The total value of money of final goods and services produced by a country in a year is:
  \begin{choices}
  \choice GDP
  \choice NNP
  \choice GNP
  \choice NI
  \end{choices}

\question High value commodities production priority zone in APP is:
  \begin{choices}
  \choice Terai and inner terai
  \choice Hill and mountain
  \choice Inner himalayan region
  \choice Terai and mid-hills
  \end{choices}

\question The main obstacle in agricultural marketing in Nepal is:
  \begin{choices}
  \choice Lack of price policy
  \choice Lack of institutional marketing
  \choice Marketing law
  \choice Middleman
  \end{choices}

\question The contribution of potato crop in AGDP of Nepal is:
  \begin{choices}
  \choice 1.4\%
  \choice 4.4\%
  \choice 6.4\%
  \choice 9.4\%
  \end{choices}

\question Which of the following is not a soil fumigant ?
  \begin{choices}
  \choice DD mixture
  \choice Nemagon
  \choice Zireb
  \choice Vapam
  \end{choices}

\question Disease which occurs occassionally by in very severe form is called:
  \begin{choices}
  \choice Endemic
  \choice Epidemic
  \choice Sporadic
  \choice Pandemic
  \end{choices}

\question Whiptail disease is caused by the deficiency of:
  \begin{choices}
  \choice Iron
  \choice Iodine
  \choice Molybdenum
  \choice Sodium
  \end{choices}

\question Which one of the following is a herbicide:
  \begin{choices}
  \choice Mancozeb
  \choice Atrazine
  \choice Cytokinin
  \choice Furadane
  \end{choices}

\question Chemical fertilizers were first introduced in Nepal in:
  \begin{choices}
  \choice 1947
  \choice 1952
  \choice 1967
  \choice 1977
  \end{choices}

\question Which one of the following soil possesses high water holding capacity ?
  \begin{choices}
  \choice Sandy
  \choice Loam
  \choice Clay loam
  \choice Sandy loam
  \end{choices}

\question Zinc plays vital role in:
  \begin{choices}
  \choice DNA production
  \choice Respiration
  \choice Osmosis
  \choice Photosynthesis
  \end{choices}

\question Salt tolerant species of plants are called:
  \begin{choices}
  \choice Mesophytes
  \choice Xerophytes
  \choice Halophytes
  \choice Hydrophytes
  \end{choices}

\question Which one of the following is true in case of IPM ?
  \begin{choices}
  \choice No use of pesticide at all
  \choice Judicious use of pesticides with other control methods
  \choice Using pheromones only
  \choice Organic production
  \end{choices}

\question The term horizontal revolution in agriculture refers to:
  \begin{choices}
  \choice Intensive use of all factors of production
  \choice Increased land use by utilizing marginal land
  \choice Use of high yielding varieties
  \choice Techniques of organic farming
  \end{choices}

\question Having an adverse physiological effect on survival of insect pest is called:
  \begin{choices}
  \choice Tolerance
  \choice Resistance
  \choice Antibiosis
  \choice Adoption
  \end{choices}

\question What could be the possible effects of climate change in agriculture ?
  \begin{choices}
  \choice Insect and disease outbreaks
  \choice Early ripening of crops
  \choice No seed formation in maize
  \choice All of above
  \end{choices}

\question In which of the below mentioned crops, GoN has been instantiating investment insurance ?
  \begin{choices}
  \choice Kiwi
  \choice Coffee
  \choice Tea
  \choice Potato
  \end{choices}

\question Transpiration in plants is related to \fillin[][3cm].
  \begin{choices}
  \choice Photosynthesis
  \choice Respiration
  \choice Nutrient uptake
  \choice Nutrient loss
  \end{choices}

\question The maximum permissible limit of off-type plants in foundation seed fields of cucumber crop is:
  \begin{choices}
  \choice 0.1\%
  \choice 1\%
  \choice 0.5\%
  \choice 2\%
  \end{choices}

\question Potato is \fillin[][3cm] plant.
  \begin{choices}
  \choice Monocot
  \choice Dicot
  \choice Both of above
  \choice None of above
  \end{choices}

\question \fillin[][4cm] is known as father of Green revolution.
  \begin{choices}
  \choice Dr. Abdul Kalam
  \choice Dr. Norman E. Borlaug
  \choice Einstein
  \choice Darwin
  \end{choices}

\question Which one of the following is true in case of drought problems in crops ?
  \begin{choices}
  \choice No seed formation
  \choice Dwarfing
  \choice Sterility
  \choice All of above
  \end{choices}

\question Climate change effects can be mitigated by:
  \begin{choices}
  \choice Awareness and variety development
  \choice Management
  \choice Following monitoring parameters
  \choice All of above
  \end{choices}

\question Ninja is a hybrid variety of:
  \begin{choices}
  \choice Zucchini
  \choice Cucumber
  \choice Radish
  \choice Tomato
  \end{choices}

\question Which one of the following is a major weed in rice ?
  \begin{choices}
  \choice \textit{Echinochloa colonum}
  \choice Blue mustard
  \choice \textit{Anagalis arvensis}
  \choice None of above
  \end{choices}

\end{questions}


\subsection*{\fullwidth{\Large \centering \textbf{General Knowledge Multiple Choice: Agriculture (Set 2)}}}

\begin{questions}

\question Which of the following elements is found in highest content in the urine of cattle and buffalo ?
  \begin{choices}
  \choice Nitrogen
  \choice Phosphorus
  \choice Potassium
  \choice Zinc
  \end{choices}

\question What affects the nutrient availability to a plant from the soil ?
  \begin{choices}
  \choice Soil color
  \choice Soil composition
  \choice Soil moisture content
  \choice Soil pH
  \end{choices}

\question Which among the following is used for amelerioration of acidic soil ?
  \begin{choices}
  \choice Urea
  \choice Potassium fertilizer
  \choice Agricultural lime
  \choice DAP
  \end{choices}

\question Which among the following nutrients, plants DO not obtain from the air ?
  \begin{choices}
  \choice Nitrogen
  \choice Carbon
  \choice Oxygen
  \choice All of above
  \end{choices}

\question Constitution of Nepal, 2072 has endowed farmers with which of the following fundamental rights ?
  \begin{choices}
  \choice Access to land for agriculture
  \choice Freedom of choice for cultivation of local seed
  \choice Conservation of indigenous species
  \choice All of above
  \end{choices}

\question What does food security significy for Nepal ?
  \begin{choices}
  \choice Food production at constant levels
  \choice Storage of food at constant levels
  \choice Production, distribution and consumption of necessary food commodities
  \choice Production of cash crops in required amounts
  \end{choices}

\question Civil service act, 2049 has granted \fillin[][3cm] number of days as home leave.
  \begin{choices}
  \choice 12 days
  \choice 18 days
  \choice 24 days
  \choice 30 days
  \end{choices}

\question Which among the following policies, falls under the National Agricultural Policy, 2061 ?
  \begin{choices}
  \choice Increase in agricultural production and productivity
  \choice Conservation of natural resources and environment
  \choice Development of commercial and competitive agricultural system
  \choice All of above
  \end{choices}

\question Agriculture Development Strategy, 2072 has envisioned which among the following agri-extension strategies ?
  \begin{choices}
  \choice Modern
  \choice De-centralized
  \choice Scientific
  \choice None of above
  \end{choices}

\question As of current, how many deputy directors are there under Department of Agriculture ?
  \begin{choices}
  \choice 2
  \choice 4
  \choice 6
  \choice 8
  \end{choices}

\question Excercising the rights of Article 8(Gha 2) of the Insurance Act, 2049, National Insurance Committee in coordination with Ministry of Agriculture Development issued Crop and Livestock Insurance Directives in which year ?
  \begin{choices}
  \choice 2067 BS
  \choice 2068 BS
  \choice 2069 BS
  \choice 2070 BS
  \end{choices}

\question Which among the following the major problem of agriculture development in Nepal ?
  \begin{choices}
  \choice Low investment in agriculture sector
  \choice Increasing labor out-migration
  \choice Lack of infrastructures required for agriculture
  \choice All of above
  \end{choices}

\question Apple is a \fillin[][3cm] crop.
  \begin{choices}
  \choice Tropical
  \choice Sub-tropical
  \CorrectChoice Temperate
  \choice Evergreen
  \end{choices}

\question Which region of the country is suited for seed production of cauliflower ?
  \begin{choices}
  \CorrectChoice High hills
  \choice Mid hills
  \choice Terai
  \choice All of above
  \end{choices}

\question Mango is primarily \fillin[][3cm] crop.
  \begin{choices}
  \choice Temperate region
  \choice Tropical region
  \choice Both A and B
  \choice None of above
  \end{choices}

\question Which among the following crops does Khumal Laxmi variety belong to ?
  \begin{choices}
  \CorrectChoice Potato
  \choice Pumpkin
  \choice Tomato
  \choice Brinjal
  \end{choices}

\question Hayward is the improved variety of \fillin[][3cm].
  \begin{choices}
  \choice Pomegranete
  \choice Perssimmon
  \CorrectChoice Kiwi
  \choice Walnut
  \end{choices}

\question Which among the following is the cheapest method of off-season vegetable production in Nepal ?
  \begin{choices}
  \choice Utilization of geographical diversity
  \choice Plastic house cultivation
  \choice Green house cultivation
  \choice Adoption of modern technology
  \end{choices}

\question What is the major reason for decline of citrus fruits in Nepal ?
  \begin{choices}
  \choice Citrus greening
  \choice Fruit fly
  \choice Root rot
  \choice Canker
  \end{choices}

\question Which among the following is the agriculture extension education ?
  \begin{choices}
  \choice Formal
  \choice Informal
  \choice Non-continuous
  \choice Technical
  \end{choices}

\question A group of how many members is considered suitable for primary farmer's group ?
  \begin{choices}
  \choice 20-25
  \choice 40-45
  \choice 5-10
  \choice 50-55
  \end{choices}

\question T and V system was first implemented in which district in Nepal ?
  \begin{choices}
  \choice Chitwan, Makawanpur
  \choice Bara, Parsa
  \choice Rupandehi, Nawalparasi
  \choice Jhapa, Morang
  \end{choices}

\question What is the responsibility of Agri-extension agent ?
  \begin{choices}
  \choice Method demonstration
  \choice Result demonstration
  \choice Informing farmers of new technologies
  \choice All of above
  \end{choices}

\question With respect to adoption of technology and cost effectiveness, which among the following extension approaches is considered most effective ?
  \begin{choices}
  \choice Personal contact
  \choice Group approach
  \choice Mass communication
  \choice All of above
  \end{choices}

\question Which category of farmers adopt a new technology the quickest ?
  \begin{choices}
  \choice Early majority
  \CorrectChoice Innovator
  \choice Early adoptor
  \choice Laggard
  \end{choices}

\question The word "Agronomy" is derived from \fillin[][3cm].
  \begin{choices}
  \choice English
  \choice Japanese
  \choice Latin
  \choice Greek
  \end{choices}

\question Which class of seed has a tag of yellow background with letters printed in black color ?
  \begin{choices}
  \choice Breeder seed
  \choice Foundation seed
  \CorrectChoice Improved seed
  \choice All of above
  \end{choices}

\question The scientific name of Pigeon pea is \fillin[][3cm].
  \begin{choices}
  \choice Vigna mungo
  \choice Glycine max
  \CorrectChoice Cajanus cajan
  \choice Vicia faba
  \end{choices}

\question Khajura Durum-2 is a variety \fillin[][3cm].
  \begin{choices}
  \choice Lentil
  \choice Rice
  \CorrectChoice Wheat
  \choice Maize
  \end{choices}

\question Fingermillet is a crop of family \fillin[][3cm].
  \begin{choices}
  \choice Leguminosae
  \CorrectChoice Poaceae
  \choice Cucurbitaceae
  \choice Cruciferae
  \end{choices}

\question Which variety of maize among the ones listed below mature earliest ?
  \begin{choices}
  \choice Arun-3
  \choice Ganesh-2
  \choice Manakamana-7
  \choice Poshilo Makai-2
  \end{choices}

\question Radha-14 is a variety of \fillin[][3cm].
  \begin{choices}
  \choice Maize
  \choice Fingermillet
  \choice Wheat
  \CorrectChoice Rice
  \end{choices}

\question IPM technique, which aims for wholesome management of crop pests was first initiated in Nepal in \fillin[][3cm].
  \begin{choices}
  \choice 1960 AD
  \choice 1977 AD
  \choice 1987 AD
  \CorrectChoice 1997 AD
  \end{choices}

\question What does the red colored labelling in pesticide container indicate ?
  \begin{choices}
  \CorrectChoice Extremely hazardous
  \choice Hazardous
  \choice Moderately hazardous
  \choice Slightly hazardous
  \end{choices}

\question Bordeaux mixture is made of \fillin[][3cm].
  \begin{choices}
  \CorrectChoice Copper sulphate
  \choice Sodium chloride
  \choice Calcium
  \choice All of above
  \end{choices}

\question Chlorpyriphos is a \fillin[][3cm].
  \begin{choices}
  \choice Fungicide
  \CorrectChoice Insecticide
  \choice All of above
  \choice Herbicide
  \end{choices}

\question What is the training duration required for obtaining pesticide-retail trading license ?
  \begin{choices}
  \choice 2 days
  \choice 4 days
  \choice 6 days
  \choice 8 days
  \end{choices}

\question Neck blast is the disease of \fillin[][3cm].
  \begin{choices}
  \choice Sugarcane
  \CorrectChoice Rice
  \choice Wheat
  \choice Mustard
  \end{choices}

\question Which pheromone is useful for the control of Fruit fly of cucurbits ?
  \begin{choices}
  \choice Methyl eugenol
  \choice Cu-lure
  \choice Heli-lure
  \choice All of above
  \end{choices}

\question Which among following the markets characterize agriculture sector markets of Nepal ?
  \begin{choices}
  \choice Group market
  \choice Hat bazzar
  \choice Cooperative market
  \choice All of above
  \end{choices}

\question Which among the following are the components of agricultultural market system ?
  \begin{choices}
  \choice Middlemen/broker
  \choice Collectors
  \choice Producers
  \CorrectChoice All of above
  \end{choices}

\question Which among the following definitions best describes "Hat bazzar" ?
  \begin{choices}
  \choice A market assigned to specific location at specific day and time
  \choice It is also called a rural market
  \choice It enables direct contact of consumers and producers
  \choice All of above
  \end{choices}

\question What is market-oriented agriculture system ?
  \begin{choices}
  \choice An agricultural system which produces based on market demands.
  \choice Export-oriented agricultural system
  \choice An agricultural system which produces based on demand of wholesale market
  \choice An agricultural system which produces based on demand of retail market
  \end{choices}

\question Farm gate price refers to \fillin[][3cm].
  \begin{choices}
  \choice Market price
  \choice Retail price
  \CorrectChoice Price obtained by farmers
  \choice Price paid by consumers
  \end{choices}

\question What is primary data ?
\begin{choices}
\CorrectChoice Unprocessed and unpublished data
\choice Processed and published data
\choice Both of above
\choice None of above
\end{choices}

\question While taking a crop cutting of rice in 10 m x 10 m, 100 kg of production was obtained. What is the production of rice (in hectares) ?
\begin{choices}
\choice 10 mt
\choice 5 mt
\choice 20 mt
\choice 25 mt
\end{choices}

\question Which among the following are the principle of organic agriculture ?
\begin{choices}
\choice Principle of ecology
\choice Principle of health
\choice Principle of soil management
\choice All of above
\end{choices}

\question What is the appropriate depth to be sampled while obtaining soil sample for cereal crops ?
\begin{choices}
\choice Upper 10 cm
\choice Upper 15 cm
\choice Upper 20 cm
\choice Upper 30 cm
\end{choices}

\end{questions}


\subsection*{\fullwidth{\Large \centering \textbf{General Knowledge Multiple Choice: Agricultural Economics and Farm Management (Set 2)}}}

\begin{questions}

\question Wind is the example of \fillin[][3cm] ?
\begin{choices}
\choice Flow resource
\choice Stock resource
\choice Renewable resource
\choice Both I and III
\end{choices}

\question \fillin[][3cm] is also called Mini cycle within the project cycle.
\begin{choices}
\choice Identification
\choice Preparation and analysis
\choice Appraisal
\choice Implementation
\end{choices}

\question The project plan may cost up to \fillin[][3cm]?
\begin{choices}
\choice 7-10\% of total investment
\choice 20-30\% of total investment
\choice 50\% of total investment
\choice None
\end{choices}

\question In \fillin[][3cm] shadow price or accounting price are used.
\begin{choices}
\choice Economic analysis
\choice Financial analysis
\choice Commercial aspect
\choice Technical aspect
\end{choices}

\question Which is the correct formula of calculation of IRR ?
\begin{choices}
\choice IRR=LDR+D(NPV at UDR)/sum of NPV at TDRs
\choice IRR=UDR+D(NPV at LDR)/sum of NPV at TDRs
\choice IRR=LDR+D(NPV at LDR)/sum of NPV at TDRs
\choice None
\end{choices}

\question \fillin[][3cm] is true type of externality.
\begin{choices}
\choice Pecuniary externality
\choice Technical externality
\choice Both
\choice None
\end{choices}

\question In \fillin[][3cm] people are asked directly to report their willingness to pay (WTP) to obtain a specified good, or willingness to accept (WTA) to give up a good.
\begin{choices}
\choice Hedonic pricing method
\choice Contingent valuation method
\choice Opportunity cost method
\choice Travel cost method
\end{choices}

\question Which is the process of determination of value or worth of a project ?
\begin{choices}
\choice Planning
\choice Monitoring
\choice Evaluation
\choice None
\end{choices}

\question The evaluation conducted during the implementation phase of program or project is \fillin[][3cm].
\begin{choices}
\choice Pre-evaluation
\choice Ongoing evaluation
\choice Terminal evaluation
\choice Ex-post evaluation
\end{choices}

\question Forest and grassland covers \fillin[][3cm] area of Nepal in 2017/18.
\begin{choices}
\choice 39\%
\choice 40\%
\choice 42\%
\choice 44\%
\end{choices}

\end{questions}


\subsection*{\fullwidth{\Large \centering \textbf{General Knowledge Multiple Choice: Agriculture (Set 4)}}}

\begin{questions}

\question Blue revolution is about ?
\begin{items}
\item* Fish production
\item Petroleum and biodiesel prodution
\item Tomato and meat production
\item Egg and poultry production
\end{items}

\question Egg and poultry are related to ?
\begin{items}
\item Golden revolution
\item Red revolution
\item* Silver revolution
\item Grey revolution
\end{items}

\question Breeder seed has \fillin[][3cm]?
\begin{items}
\item White tag
\item White tag with blue border
\item Yellow tag
\item* Brown tag
\end{items}

\question 1 kattha of land is \fillin[][3cm] square meters.
\begin{items}
\item 182.25
\item* 338.63
\item 508.72
\item 256
\end{items}

\question \fillin[][2.5cm] is entitled vegetable meat.
\begin{items}
\item Brinjal
\item Cowpea
\item Bitter gourd
\item Tomato
\end{items}

\question \fillin[][3cm] deficiency causes dieback of shoots.
\begin{items}
\item Boron
\item* Copper
\item Molybdenum
\item Zinc
\end{items}

\question Which among the following is the most cultivated rice variety in Nepal ?
\begin{items}
\item Ghaiya
\item Radha-14
\item* Sona mansuli
\item Chaite
\end{items}

\question Which fruit is called a miracle fruit ?
\begin{items}
\item Mandarin
\item Banana
\item Kiwi
\item Avocado
\end{items}

\question Chekurmanis (\textit{Sauropus androgynus}) is a \fillin[][3cm].
\begin{items}
\item Pteridophyte and vegetable
\item* Angiosperm and vegetable
\item Gymnosperm and vegetable
\item 20th century vegetable
\end{items}

\question Oleoresin is extracted from ?
\begin{items}
\item Papaya
\item* Chillies
\item Jackfruit
\item Brinjal
\end{items}

\question Chaubatia paste is made in \fillin[][3cm].
\begin{items}
\item Water
\item* Linseed oil
\item Kerosine
\item Palm oil
\end{items}

\question Carbamates are \fillin[][3cm].
\begin{items}
\item* Sulfur fungicides
\item Mercurial insecticides
\item Alkylating agents
\item Chlorinated hydrocarbon
\end{items}

\question Angle of repose of paddy is \fillin[][3cm].
\begin{items}
\item* 30-45 degrees
\item 23-28 degrees
\item 20-30 degrees
\item 15-20 degrees
\end{items}

\question Cellulose is absent in \fillin[][3cm].
\begin{items}
\item Primay cell wall
\item* Middle lamella
\item Secondary cell wall
\item Tertiary cell wall
\end{items}

\question Which technology is used to produce monoclonal antibodies ?
\begin{items}
\item* Hybridoma
\item Antibody fragmentation
\item Serial dilution
\item Conjugation
\end{items}

\question What is the composition of bordeaus mixture ?
\begin{items}
\item 10 lb CusSO4 + 5 lb Lime + 100 gallon water
\item 5 lb CuSO4 + 5 lb Lime + 100 gallon water
\item 10 lb CuSO4 + 10 lb Lime + 100 gallon water
\item* 5 lb CuSO4 + 5 lb Lime + 50 gallon water
\end{items}

\question Which of the following disasters does crop insurance does not cover ?
\begin{items}
\item Biotic (insect and disease) damage
\item Fire
\item Frost and hail
\item* Non-germination
\end{items}

\question \fillin[][3cm] percentage of population of Gandaki province are engaged in agriculture.
\begin{items}
\item* 64
\item 60
\item 66
\item 72
\end{items}

\question Recently government has announced subsidization on transportation fare in of perishable commodities (including cereals) from production site to nearby market. What percentage of transport fees is provisioned for subsidy ?
\begin{items}
\item* 25 percent
\item 20 percent
\item 30 percent
\item 40 percent
\end{items}

\question Government imposed nationwide lockdown from \fillin[][3cm] to prevent and control COVID-19 pandemic.
\begin{items}
\item March 20
\item April 8
\item* March 24
\item April 3
\end{items}

\question According to the Crops and livestock Insurance Directives, 2013 (2070), government has announced \fillin[][3cm] per cent subsidy in premium as agriculture insurance to farmers.
\begin{items}
\item 75 percent
\item* 50 percent
\item 85 percent
\item 70 percent
\end{items}

\question ADS is in implementation since \fillin[][3cm] year plan.
\begin{items}
\item* 13 th
\item 14 th
\item 12 th
\item 11 th
\end{items}

\question In which schedule of the Constitution of Nepal is there the provision of combined right to agriculture sector ?
\begin{items}
\item Schedule 7
\item Schedule 8
\item* Schedule 9
\item Schedule 10
\end{items}

\question Which of the following diseases is also called The Yellow Dragon disease.
\begin{items}
\item Citrus gall
\item Apple scab
\item* Citrus greening
\item Mango malformation
\end{items}

\question Ballot box test is a useful approach in Farmers' field school to \fillin[][3cm].\begin{items}
\item* To understand what farmers know and if they have any gap in knowledge
\item To promote entertainment
\item To provide subsidy for farmers
\item To access environmental status
\end{items}

\question Which among the following crops does the grain weevil damages the most ?
\begin{items}
\item* Maize (?)
\item Wheat
\item Rice
\item Lentil
\end{items}

\question Province 4 is not self-sufficient in \fillin[][3cm].
\begin{items}
\item Fruit
\item Vegetable
\item Maize
\item* Rice
\end{items}

\question According to Vision 2076-2100 of the Gandaki province, what is the productivity goal of major agricultural commodities by the end of Long term plan period ?
\begin{items}
\item 2.8 ton per hectare
\item 4 ton per hectare
\item 5 ton per hectare
\item* 6 ton per hectare
\end{items}

\question Which among the following chemical pesticide(s) binds with Acetylcholinesterase and inhibits its activity ?
\begin{items}
\item Carbamates
\item Organophosphates
\item* Both of above
\item None of above
\end{items}

\question What amount of land (in hectares) is required to plant 10 square kilometers of apple ?
\begin{items}
\item 10 ha
\item 100 ha
\item* 1000 ha
\item 10000 ha
\end{items}

\question What is the consequence of use of organic fertilizer in Grape and Sugarcane ?
\begin{items}
\item Increases pigmentation
\item All of above
\item* Increases sugar content
\item None of above
\end{items}

\question Fruit cracking is a serious problem in \fillin[][3cm].
\begin{items}
\item Walnut
\item Banana
\item Apple
\item* Litchi
\end{items}

\question Which orchard first started apple cultivation in Nepal ?
\begin{items}
\item Jumla farm
\item* Putalibagaicha of Kathmandu
\item Marpha farm of Mustang
\item Daman farm
\end{items}

\question Goal of extension education is \fillin[][3cm].
\begin{items}
\item To promote income of the farmer
\item To promote new crop
\item To promote production of the crop
\item* To promote scientific outlook
\end{items}

\question Which among the following approach of extension is used in participating rural youths ?
\begin{items}
\item Tuki
\item Group
\item* Charpatey club
\item All of above
\end{items}

\question Which approach of extension relies on training of extension agent by scientist of research station ?
\begin{items}
\item Tuki
\item Block production
\item* Training and Visit
\item None of above
\end{items}

\question What is the purpose of agrovet establishment in Nepal ?
\begin{items}
\item Sales of agricultural products
\item Sales of agricultural inputs
\item Communication of information on agriculture
\item* All of above (?)
\end{items}

\question When does the agriculture program broadcast in TV and radio Nepal ?
\begin{items}
\item 6:30 pm
\item 6:40 am
\item 6:45 am
\item* None of above
\end{items}

\question Which among the following is not a stage in Adoption of Innovation ?
\begin{items}
\item Evaluation
\item Trialing
\item Interest
\item* Need identification
\end{items}

\question Deficiency of which micronutrient causes little leaf of cotton ?
\begin{items}
\item Boron
\item Nitrogen
\item Calcium
\item* Zinc
\end{items}

\question Which nutrient becomes available at excess in acidic soil thereby causing toxicity in plants ?
\begin{items}
\item Calcium
\item Magnesium
\item Both
\item* Aluminium
\end{items}

\question Organically prepared biochar has highest content of \fillin[][3cm].
\begin{items}
\item Nitrogen
\item Potassium
\item Phosphorus
\item* Carbon
\end{items}

\question Which among the following elements is not essential nitrient of plants ?
\begin{items}
\item Boron
\item Hydrogen
\item Magnesium
\item* Sodium
\end{items}

\question Use of which of the following is a method of pH detection of soil ?
\begin{items}
\item Munsell color chart
\item* Litmus paper
\item Both
\item None of above
\end{items}

\question Which among the following crops requires highest amount of water ?
\begin{items}
\item Rice
\item Wheat
\item Maize
\item* Sugarcane
\end{items}

\question In an area where both Maize and Soybean can be grown together, which pattern of cultivation relies on planting maize exclusively without inter-space sowing of soybean ?
\begin{items}
\item Mixed cropping
\item* Monocropping
\item Relay cropping
\item Intercropping
\end{items}

\question Which among the following is a cash crop ?
\begin{items}
\item Sugarcane
\item Oilseeds
\item Potato
\item* All
\end{items}

\question Which among the following is a popular Rice variety in Kathmandu valley ?
\begin{items}
\item Palung dhan
\item* Khumal-4
\item Hardinath-2
\item Radha-12
\end{items}

\question Which region of Nepal is suitable for Jute cultivation ?
\begin{items}
\item Central terai
\item Western terai
\item Far-western terai
\item* Eastern terai
\end{items}

\question Why is rouging done in crop seed production ?
\begin{items}
\item To increase germination rate
\item To maintain genetic purity
\item To remove small plants
\item* To maintain physical purity (?)
\end{items}

\question What is/are the alternative name of Amaranthus (\textit{lattey}) in Nepal ?
\begin{items}
\item Ramdana
\item Lunde and Marshi
\item Bet or Chaulai
\item* All
\end{items}

\question Dual track management system (DTMS) has been launched in how many of Agriculture farms in Nepal initially ?
\begin{items}
\item 2
\item 1
\item 3
\item* 4
\end{items}

\question Budwood certification system has been started in which of the following fruit crop ?
\begin{items}
\item Mango
\item Temperate fruits
\item* Citrus fruits
\item None of above
\end{items}

\question Which of the following methods of IPM relies on use of an attractive lure in alluring insect-pests ?
\begin{items}
\item Biological
\item* Physical (?)
\item Both
\item None of above
\end{items}

\question Which of the following agricultural crops is a perishable commodity ?
\begin{items}
\item* Leafy vegetable
\item Pulses
\item Both
\item None
\end{items}

\question Farmers popularly acquire \fillin[][3cm] loan in agri-business operation ?
\begin{items}
\item* Unorganized institutions
\item Organized institutions
\item Nothing can be said
\item Farmers do not acquire loan
\end{items}

\question Which among the following aspects is most important in market management ?
\begin{items}
\item* Roadway
\item Production
\item Storage
\item Ropeway
\end{items}

\question Kalimati Fruit and Vegetable Market is established with \fillin[][3cm] legal provision.
\begin{items}
\item* Development committee act
\item Market act
\item Market committee act
\item Local market act
\end{items}

\question What is the consequence of increased supply in the market in price?
\begin{items}
\item* Price decreases
\item Price increases
\item Price does not change
\item Price fluctuates
\end{items}

\question Due to use of Calcium Carbide in Apple, Mango and alike crop, which of the following processes is hastened ?
\begin{items}
\item Reduction in sourness
\item* Promotes ripening
\item Control of fungal disease
\item Reduction in fruit drop
\end{items}

\question How was Kinnow mandarin developed ?
\begin{items}
\item Hybridization of multiple species of citrus
\item Selection of high yielding Sweet orange
\item* Crossing between Mandarin and Sweet orange
\item Vegetative propagation of high yielding Sweet orange
\end{items}

\end{questions}


\subsection*{\fullwidth{\Large \centering \textbf{General Knowledge Multiple Choice: Agriculture (Set 5)}}}

\begin{questions}

\question Ambient relative humidty for the growth of Oyster mushroom is \fillin[][3cm] percent.
\begin{choices}
\CorrectChoice 85
\choice 75
\choice 80
\choice 90
\end{choices}

\question Which crop acts as a trap for control of Nematodes ?
\begin{choices}
\CorrectChoice Marigold
\choice Neem
\choice Soybean
\choice Cucumber
\end{choices}

\question Wart disease is seen in
\begin{choices}
\choice Cabbage
\choice Maize
\CorrectChoice Potato
\choice Tomato
\end{choices}

\question Which fungicide is best suited to control powdery mildew of rose ?
\begin{choices}
\choice Dithane M45
\choice Blitox 50
\choice Thiram
\end{choices}
\CorrectChoice Sulfex 80 WP

\question Which part of the cauliflower does the knot disease due to \textit{Plasmodiophora brassicae} affects ?
\begin{choices}
\choice Curd
\CorrectChoice Main root
\choice Leaf
\choice Stem
\end{choices}

\question Plant protection action was implemented in year \fillin[][3cm] AD.
\begin{choices}
\choice 2010
\CorrectChoice 2007
\choice 2002
\choice 2000
\end{choices}

\question How many chemical pesticides are currently banned in Nepal ?
\begin{choices}
\choice 26
\choice 16
\CorrectChoice 24
\choice 12
\end{choices}

\question What is/are the main reason for lagging production of Jute, Cotton and Tobaccoo in Nepal ?
\begin{choices}
\choice Unfavorable climate
\choice Lack of labor
\choice Lack of experience of farmers
\CorrectChoice Poor development of industrial infrastructure
\end{choices}

\question What is the seed rate of Arun-7 maize variety ?
\begin{choices}
\CorrectChoice 20
\choice 40
\choice 50
\choice 10
\end{choices}

\question Bahuguni-1 and Bahuguni-2 rice varieties are best suited in \fillin[][3cm] cropping pattern.
\begin{choices}
\choice Rice-Wheat
\choice Rice-Maize
\choice Rice-Lentil
\CorrectChoice All
\end{choices}

\question Which one of the following wheat varieties released the latest ?
\begin{choices}
\CorrectChoice Banganga
\choice Gautam
\choice Annapurna
\choice Pshanglhamu
\end{choices}

\question What is the color of the tag of the seed of Improved class ?
\begin{choices}
\choice White
\CorrectChoice Yellow
\choice Green
\choice Blue
\end{choices}

\question The productivity of Mithe fapar variety of Buckwheat is \fillin[][3cm] mt per ha.
\begin{choices}
\CorrectChoice 1.4
\choice 2
\choice 4
\choice 5
\end{choices}

\question What is the most important factor in Adoption of innovation ?
\begin{choices}
\choice Source
\choice Message
\CorrectChoice Adopter
\choice Individual communicating the message
\end{choices}

\question What is the goal of an active farmer's group ?
\begin{choices}
\CorrectChoice Resolve common problem
\choice Solve personal problem
\choice Organize meeting and disburse allowance
\choice All of above
\end{choices}

\question How can the training be made effective ?
\begin{choices}
\choice Talking and Seeing
\CorrectChoice Talking, Seeing, and Doing
\choice Talking only
\choice Seeing only
\end{choices}

\question Recent popular method of agriculture extension and services are done through,
\begin{choices}
\choice School children participation
\CorrectChoice Farmer's group approach
\choice Individual farmer contact
\choice Political leaders
\end{choices}

\question The medium of communication in which steps involved in a new technology is shown pictorially for visualization is
\begin{choices}
\CorrectChoice Flip chart
\choice Flannel board
\choice Still picture
\choice Leaflet
\end{choices}

\question What needs to be considered while planning local level agriculture program ?
\begin{choices}
\choice Local needs
\choice Local potential
\choice National policy
\CorrectChoice All of above
\end{choices}

\question What crop is "California" variety related to ?
\begin{choices}
\CorrectChoice Chilly
\choice Cauliflower
\choice Tomato
\choice Okra
\end{choices}

\question Citrus greening is caused by
\begin{choices}
\choice Citrus aphid
\CorrectChoice Citrus psylla
\choice Citrus scale
\choice Citrus fruit fly
\end{choices}

\question What is the seed to seed spacing while planting coffee seed for seedling production ?
\begin{choices}
\choice 5-6 cm
\choice 7-8 cm
\CorrectChoice 3-5 cm
\choice 8-10 cm
\end{choices}

\question "President Best Performing Farmer Award" has been intitiated since \fillin[][3cm] to dignify agriculture occupation.
\begin{choices}
\CorrectChoice 2071 BS
\choice 2068 BS
\choice 2061 BS
\choice 2058 BS
\end{choices}

\question When was National Agriculture Policy implemented ?
\begin{choices}
\choice 2058 BS
\CorrectChoice 2061 BS
\choice 2059 BS
\choice 2063 BS
\end{choices}

\question Which among the following NGOs is involved in varietal development and conservation activities.
\begin{choices}
\CorrectChoice Local Initiatives for Biodiversity Research and Development
\choice Tamakoshi Development Committee
\choice All-round Development Society
\choice Cooperative Rural Development Society
\end{choices}

\question Which of the following agricultural products is "Low volume, high value" commodity ?
\begin{choices}
\choice Fingermillet
\choice Rice
\choice Maize
\CorrectChoice Fish
\end{choices}

\question
\begin{choices}
\CorrectChoice
\choice
\choice
\choice
\end{choices}

\question
\begin{choices}
\CorrectChoice
\choice
\choice
\choice
\end{choices}

\question
\begin{choices}
\CorrectChoice
\choice
\choice
\choice
\end{choices}

\question
\begin{choices}
\CorrectChoice
\choice
\choice
\choice
\end{choices}

\question
\begin{choices}
\CorrectChoice
\choice
\choice
\choice
\end{choices}

\question
\begin{choices}
\CorrectChoice
\choice
\choice
\choice
\end{choices}

\question
\begin{choices}
\CorrectChoice
\choice
\choice
\choice
\end{choices}

\question
\begin{choices}
\CorrectChoice
\choice
\choice
\choice
\end{choices}

\question
\begin{choices}
\CorrectChoice
\choice
\choice
\choice
\end{choices}

\question
\begin{choices}
\CorrectChoice
\choice
\choice
\choice
\end{choices}

\question
\begin{choices}
\CorrectChoice
\choice
\choice
\choice
\end{choices}

\question
\begin{choices}
\CorrectChoice
\choice
\choice
\choice
\end{choices}

\question
\begin{choices}
\CorrectChoice
\choice
\choice
\choice
\end{choices}

\question
\begin{choices}
\CorrectChoice
\choice
\choice
\choice
\end{choices}

\question
\begin{choices}
\CorrectChoice
\choice
\choice
\choice
\end{choices}

\question
\begin{choices}
\CorrectChoice
\choice
\choice
\choice
\end{choices}

\question
\begin{choices}
\CorrectChoice
\choice
\choice
\choice
\end{choices}

\question
\begin{choices}
\CorrectChoice
\choice
\choice
\choice
\end{choices}

\question
\begin{choices}
\CorrectChoice
\choice
\choice
\choice
\end{choices}

\question
\begin{choices}
\CorrectChoice
\choice
\choice
\choice
\end{choices}

\question
\begin{choices}
\CorrectChoice
\choice
\choice
\choice
\end{choices}

\question
\begin{choices}
\CorrectChoice
\choice
\choice
\choice
\end{choices}

\question
\begin{choices}
\CorrectChoice
\choice
\choice
\choice
\end{choices}

\question
\begin{choices}
\CorrectChoice
\choice
\choice
\choice
\end{choices}

\question
\begin{choices}
\CorrectChoice
\choice
\choice
\choice
\end{choices}

\question
\begin{choices}
\CorrectChoice
\choice
\choice
\choice
\end{choices}

\question
\begin{choices}
\CorrectChoice
\choice
\choice
\choice
\end{choices}

\end{questions}


\end{document}
