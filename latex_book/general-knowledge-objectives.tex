\section*{\fullwidth{\large \centering \textbf{General Knowledge: Objective}}}

\subsection*{\fullwidth{\Large \centering \textbf{Multiple choice}}}

\begin{questions}

%% in order to provide indication for marks of each question, do it as follows:
% \question[1] What is the relationship between latitude and temperature ?
\question What is the relationship between latitude and temperature ?
  \begin{enumerate}
  \item With increase in latitude, temperature increases
  \item With increase in latitude, temperature decreases
  \item With decrease in latitude, temperature increases
  \item With decrease in latitude, temperature decreases
  \end{enumerate}

  \begin{choices}
  \choice 1 and 3 are correct
  \choice 1 and 2 are correct
  \CorrectChoice 2 and 3 are correct
  \choice 3 and 4 are correct
  \end{choices}

\question Large and bright meteors are called:
  % this places all choices in separate lines
  \begin{choices}
  \choice Falling star
  \choice Shooting star
  \CorrectChoice Fire ball
  \choice Shooting ball
  \end{choices}

\question The country with most languages spoken is:
  % this places all choices in the same line

  \begin{oneparchoices}
  \choice China
  \CorrectChoice India
  \choice America
  \choice Russia
  \end{oneparchoices}

\question Which is the biggest of the deserts ?

  \begin{oneparchoices}
  \choice Somali
  \CorrectChoice Karakoram
  \choice Thar
  \choice Attacama
  \end{oneparchoices}

\question Popular piligrimage Muktinath occurs in the elevation of:
  \begin{choices}
  \choice 4500 m
  \CorrectChoice 3750 m
  \choice 3500 m
  \choice 4100 m
  \end{choices}

\question Which amongst these nations have been elected in UN security council for most number of times?
  \begin{choices}
  \choice Japan
  \choice Spain
  \choice Nepal
  \CorrectChoice Uruguay
  \end{choices}

\question Chess is the national game of:

  \begin{oneparchoices}
  \choice India
  \CorrectChoice Russia
  \choice Saudi Arabia
  \choice none of above
  \end{oneparchoices}

\question Nepal is not the earliest among south asian nations in:
  \begin{choices}
  \choice Maintaining diplomatic policy with Israel
  \CorrectChoice Incorporating "Rights to information" in constitution
  \choice Formulating a separate law regarding "Rights to information"
  \choice Summiting Mt. Everest.
  \end{choices}

\question Which among the below is known as the city of handcrafts:
  \begin{choices}
  \choice Bhaktapur
  \CorrectChoice Patan
  \choice Bungmati
  \choice Bandipur
  \end{choices}

\question Nepal is the \fillin[][2cm] nation to have issued constitution through a constitutional assembly.
  \begin{choices}
  \choice 28
  \choice 35
  \CorrectChoice 44
  \choice 53
  \end{choices}

\question Which country among the following is the first to grant voting rights to women ?
  \begin{choices}
  \choice America
  \choice Greece
  \CorrectChoice New Zealand
  \choice Japan
  \end{choices}

\question Nepal started elephant polo completion in the year:
  \begin{choices}
  \choice 1962 AD
  \CorrectChoice 1982 AD
  \choice 1992 AD
  \choice 2002 AD
  \end{choices}

\question Which among the following cities of Nepal is the first "Eco" city:
  \begin{choices}
  \choice Pokhara
  \choice Madi
  \choice Dharan
  \CorrectChoice Bharatpur
  \end{choices}

\question Which year did the SAARC university started to operate in
  \begin{choices}
  \choice 2008
  \choice 2009
  \CorrectChoice 2010
  \choice 2011
  \end{choices}

\question In how many countries of the world tigers are found?
  \begin{choices}
  \choice 5
  \choice 9
  \CorrectChoice 13
  \choice 14
  \end{choices}

\question When did EuroCup started ?
  \begin{choices}
  \CorrectChoice 1960
  \choice 1964
  \choice 1968
  \choice 1990
  \end{choices}

\question Which among the following cities does not comprise the Silk road?
  \begin{choices}
  \choice Bukhara, Uzbekistan
  \choice Kabul, Afganistan
  \CorrectChoice Jihadh, Saudi Arabia
  \choice Aleppo, Syria
  \end{choices}

\question ASEAN has how many member and supervisor nations?
  \begin{choices}
  \CorrectChoice 10, 2
  \choice 9, 3
  \choice 10, 4
  \choice 10, 5
  \end{choices}

\question The tallest peak of Chure range - Garba - has elevation of:
  \begin{choices}
  \CorrectChoice 1872 m
  \choice 1730 m
  \choice 1827 m
  \choice 1890 m
  \end{choices}

\question "It is our world" is the slogan of \fillin[][3cm] organization:
  \begin{choices}
  \choice WB
  \choice WHO
  \CorrectChoice UNO
  \choice All of above
  \end{choices}

\question BP Koirala inaugurated the BP highway in:
  \begin{choices}
  \CorrectChoice Shreekhandapur, Kavre
  \choice Dumla, Sindhuli
  \choice Bardibas, Mahottari
  \choice Sindhuligadi, Sindhuli
  \end{choices}

\question Nepal's first civil revolution is that of:
  \begin{choices}
  \CorrectChoice Biratnagar Jute mill
  \choice Jhapa revolution
  \choice Parcha revolution
  \choice None
  \end{choices}

\question BOAO forum for Asia summit organized in April 11, 2018 has elected as president:
  \begin{choices}
  \choice Xi Jin Ping
  \choice Li Khichiang
  \CorrectChoice Wan Ki Moon
  \choice Hu Jin Tao
  \end{choices}

\question Based on the office duration of individuals chairing house of representatives of Nepal which among following order is correct:
  \begin{enumerate}
  \item Ramchandra Poudel
  \item Damanath Dhungana
  \item Taranath Ranabhat
  \end{enumerate}
  \begin{choices}
  \choice 2, 3 and 1
  \CorrectChoice 2, 1 and 3
  \choice 1, 2 and 3
  \choice 3, 2 and 1
  \end{choices}

\question Banke, Bardiya, Kailali and Kanchanpur (a.k.a new nation) were returned to Nepal in:
  \begin{choices}
  \choice 1911 BS
  \choice 1915 BS
  \CorrectChoice 1917 BS
  \choice 1919 BS
  \end{choices}

\question According to village executive committee formation guidelines, in a village with 7 wards how many members form the village executive committe ?
  \begin{choices}
  \choice 13
  \choice 14
  \choice 15
  \choice 16
  \end{choices}

\question In the Human rights declaration of December 10, 1948, there are \fillin[][2cm] articles.
  \begin{choices}
  \choice 20
  \CorrectChoice 30
  \choice 32
  \choice 40
  \end{choices}

\question There are \fillin[][2cm] local bodies in Province 3.
  \begin{choices}
  \choice 116
  \choice 117
  \choice 118
  \CorrectChoice 119
  \end{choices}

\question Hulak Goshwara Adda was established during the regime of:
  \begin{choices}
  \choice Bhim Shamsher
  \CorrectChoice Chandra Shamsher
  \choice Juddha Shamsher
  \choice Dev Shamsher
  \end{choices}

\question "Rastriya Sabha Griha" was established with the assistance of \fillin[][3.5cm] while being designed by:
  \begin{choices}
  \CorrectChoice China, Gangadhar Bhatta
  \choice India, Gangadhar Bhatta
  \choice China, Atmakrishna Shrestha
  \choice India, Atmakrishna Shrestha
  \end{choices}

\question Which among the following statements are righly described below:
  \begin{enumerate}
  \item Chilime hydroelectricity project lies in Rasuwa district
  \item Chameliya hydroelectricity project lies in Darchula district
  \end{enumerate}
  \begin{choices}
  \CorrectChoice Both 1 and 2 are correct
  \choice Both 1 and 2 are wrong
  \choice 1 is correct and 2 is wrong
  \choice 2 is correct and 1 is wrong
  \end{choices}

\question 16th Aaha-Rara gold cup, 2019 finals was played between:
  \begin{choices}
  \CorrectChoice Nepal police club and Three star club
  \choice Three star club and Manang marsyangdi club
  \choice Thribhuwan army club and Nepal police
  \choice Nepal police and Sahara club
  \end{choices}

\question Which country does not have the same name for its country and capital?
  \begin{choices}
  \choice Luxemborg
  \choice San marino
  \choice Djibouti
  \CorrectChoice Sicily
  \end{choices}
  \begin{solution}
      All other are country and the names for its capital respectively with exception of Sicily which is the largest island in Mediterranean sea and one of the 20 regions of Italy.
  \end{solution}

\question Based on the area, the largest to smallest deserts of the world are:
  \begin{choices}
  \CorrectChoice Arabian, Gobi, Kalahari, Patagonia
  \choice Gobi, Arabian, Kalahari, Patagonia
  \choice Gobi, Arabian, Patagonia, Kalahari
  \choice Arabian, Gobi, Patagonia, Kalahari
  \end{choices}


\begin{solution}

Following are some of the deserts of the world. This does not enumerate the deserts by order of their sizes:

\begin{itemize}
    \item Arabian 899,618 sq miles. Spanning almost all of Arabian peninsula
    \item Atacama 600 mile long area rich in nitrate and copper deposits in northern chile
    \item Chihuahuan 139769 sq miles, Arizona and Mexico.
    \item Dasht-e Kavir 500 mile long by 200 mile wise in north-central Iran
    \item Dasht-e Lut 300 mile long by 200 mile wide in south-central Iran
    \item Death Valley 3300 sq miles, in California and Nevada
    \item Eastern (Arabian), Egypt. Between the Nile and Red sea extending south into Sudan
    \item Gibson, 60232 sq miles in the interior of western Australia
\end{itemize}

\end{solution}

\question Which of the following states does not have identity of "Baise rajya"?
  \begin{choices}
  \choice Jahari
  \choice Dullu
  \choice Bilashpur
  \CorrectChoice Khanchi
  \end{choices}

\question Which among the following matches is incorrect ?
  \begin{table}[h]
  \centering
  \begin{tabular}{lll}
    \textbf{Landmark} & & \textbf{Location} \\[2mm]
    1. Laddakh range & & Asia \\
    2. Andes range & & South america \\
    3. Alps range & & North america \\
    4. Karakoram range & & Asia \\
  \end{tabular}
  \end{table}
  \begin{choices}
  \choice 1 and 2 are incorrect
  \choice 2 and 3 are incorrect
  \choice Only 1 is incorrect
  \CorrectChoice Only 3 is incorrect
  \end{choices}

\question Which conference embodied the theme that "Women's rights are human rights"?
  \begin{choices}
  \choice The World Conference on Human Rights held by the United Nations in Vienna, Austria, on 14 to 25 June 1993.
  \choice First World Women Conference, Mexico, 1975
  \choice Second World Women Conference, Copenhegan, 1980
  \choice Fourth World Women Conference, Beijing, 1995
  \end{choices}

\question What is the correct order based on chronology of establishment ?
  \begin{choices}
  \choice Radio Nepal, Nepal Telecommunication Corporation, Gorkhapatra, Nepal Television
  \CorrectChoice Gorkhapatra, Radio Nepal, Nepal Telecommunication Corporation, Nepal Television
  \choice Nepal Telecommunication Corporation, Radio Nepal, Gorkhapatra, Nepal Television
  \choice Gorkhapatra, Nepal Telecommunication Corporation, Radio Nepal, Nepal Television
  \end{choices}

\question "For a living planet" is the motto of:
  \begin{choices}
  \choice IUCN
  \choice UNEP
  \CorrectChoice WWF
  \choice UN
  \end{choices}

\question "An agenda for peace" was presented by:
  \begin{choices}
  \CorrectChoice Butroes Butroes Gholi
  \choice Kolfi Annan
  \choice Wan Ki Moon
  \choice None of above
  \end{choices}

\question When is International day of Non-voilence observed ?
  \begin{choices}
  \CorrectChoice October 2
  \choice October 3
  \choice October 4
  \choice October 5
  \end{choices}

\question Cockpit of europe is:
  \begin{choices}
  \choice France
  \choice Germany
  \choice Serbia
  \CorrectChoice Belgium
  \end{choices}

\question Which of the following countries have not been named after river ?
  \begin{choices}
  \choice Paraguay
  \choice Jambia
  \CorrectChoice Sirya
  \choice Niger
  \end{choices}

\question Which district does not have two submetropolitan city ?
  \begin{choices}
  \choice Sunsari
  \CorrectChoice Rupandehi
  \choice Bara
  \choice Dang
  \end{choices}

\question Patagonia desert is in:
  \begin{choices}
  \choice China and congo
  \choice Chile and Argentina
  \CorrectChoice China and Kazakistan
  \choice Argentina and Brasil
  \end{choices}

\question Which of the following recognitions is also known as Nobel prize on environment
  \begin{choices}
  \choice Global 500
  \CorrectChoice Goldman environment prize
  \choice Galante conservation award
  \choice Green peace prize
  \end{choices}

\question Which among the following articles of Nepalese consitution was the first amendment made ?
  \begin{choices}
  \choice Article 43
  \choice Article 84
  \choice Article 167
  \choice Article 283
  \end{choices}

\question Katuwal lake is in:
  \begin{choices}
  \choice Okhaldhunga
  \choice Chitwan
  \CorrectChoice Kathmandu
  \choice Lamjung
  \end{choices}

\question Nagarkot, a tourism site situated in Bhaktapur, is \fillin[][2cm] masl.
  \begin{choices}
  \choice 2250
  \CorrectChoice 2175
  \choice 2275
  \choice 2165
  \end{choices}

\question Gopal dynasty had the capital in:
  \begin{choices}
  \CorrectChoice Gokarna
  \choice Matatirtha
  \choice Godabari
  \choice Bouddha
  \end{choices}

\question Folkland islands were the cause of battle in:
  \begin{choices}
  \CorrectChoice 1981
  \choice 1983
  \choice 1982
  \choice 1984
  \end{choices}

\question Which article of the constitution of Nepal contains provision of constitutional amendment ?
  \begin{choices}
  \CorrectChoice Article 274
  \choice Article 243
  \choice Article 285
  \choice Article 266
  \end{choices}

\question "Point of program" is associated with:
  \begin{choices}
  \choice Road construction
  \choice Agricultural production management
  \choice Organizational restructuring
  \CorrectChoice Foreign aid
  \end{choices}

\question "Village return" campaign was implemented in:
  \begin{choices}
  \choice 2020
  \choice 2021
  \choice 2022
  \CorrectChoice 2024
  \end{choices}

\question Jhimrukh hyrdoelectricity project is situated in:
  \begin{choices}
  \CorrectChoice Pyuthan
  \choice Salyan
  \choice Dailekh
  \choice Syangja
  \end{choices}

\question What is the size of provincial assembly in Province 7 ?
  \begin{choices}
  \CorrectChoice 53 (32 directly elected, 21 proportional)
  \choice 54 (32 directly elected, 22 proportional)
  \choice 53 (34 directly elected, 19 proportional)
  \choice 53 (30 directly elected, 34 proportional)
  \end{choices}

\question The provision regarding president and vice president that each of both should be of different geneder and of a different community is installed in constitution of Nepal's article:
  \begin{choices}
  \choice 70
  \choice 71
  \choice 72
  \CorrectChoice 73
  \end{choices}

\question Which of the following is wrong ?
  \begin{choices}
  \choice The district with 52 ponds and 53 lakes $\longrightarrow$ Kaski
  \choice The gateway of Sagarmatha $\longrightarrow$ Namchebazzar
  \choice The gateway of Nepal $\longrightarrow$ Birgunj
  \choice Switzerland of Nepal $\longrightarrow$ Jiri
  \end{choices}

\question When did the first paperless meeting of minestrial cabinet took place ?
  \begin{choices}
  \choice 2075, Shrawan 5
  \CorrectChoice 2075, Shrawan 10
  \choice 2075, Shrawan 12
  \choice 2075, Shrawan 15
  \end{choices}

\question When did Nepal signed the convention on the Elimination of all forms of Discrimination Against Women ?
  \begin{choices}
  \CorrectChoice 1991
  \choice 1992
  \choice 1993
  \choice 1994
  \end{choices}

  \begin{solution}
  The Convention on the Elimination of all Forms of Discrimination Against Women (CEDAW) is an international treaty adopted in 1979 by the United Nations General Assembly. Described as an international bill of rights for women, it was instituted on 3 September 1981 and has been ratified by 189 states.
  \end{solution}

\question Where is the UN university located ?
  \begin{choices}
  \choice America
  \choice Japan
  \CorrectChoice India
  \choice France
  \end{choices}

\question Which district is regarded as the centre of Deuki custom ?
  \begin{choices}
  \choice Achham
  \CorrectChoice Baitadi
  \choice Bajura
  \choice Dadeldhura
  \end{choices}

\question The length of Karnali bridge is:
  \begin{choices}
  \CorrectChoice 500m
  \choice 606m
  \choice 507m
  \choice 608m
  \end{choices}

\question What does honey mostly contain ?
  \begin{choices}
  \choice Protein
  \choice Vitamin B
  \CorrectChoice Carbohydrate
  \choice Sugar
  \end{choices}

\question When did Stephen Hawking die ?
  \begin{choices}
  \CorrectChoice March 14, 2018
  \choice March 15, 2018
  \choice March 16, 2018
  \choice March 17, 2018
  \end{choices}

\question Which of the facts below is true ?
  \begin{enumerate}
  \item Nepal and France tied diplomacy in 1949, April 20.
  \item France is a permanent member of UNO
  \item Nepal and France agreed in increasing support of France in civil aviation of Nepal in Falgun, 2054.
  \item Junga Bahadur Rana visited France.
  \end{enumerate}
  \begin{choices}
  \CorrectChoice All of above
  \choice 2, 3 and 4
  \choice 1, 3 and 4
  \choice 2, 1 and 4
  \end{choices}

\question For the regional alliances given below, find the correct order based on the longitude lowest to highest (East to West)
  \begin{choices}
  \choice EU, ASEAN (Association of Southeast Asian Nations), SAARC, OPEC (Organization of the Petroleum Exporting Countries)
  \CorrectChoice ASEAN, SAARC, OPEC, EU
  \choice SAARC, ASEAN, EU, OPEC
  \choice OPEC, EU, ASEAN, SAARC
  \end{choices}

\question Who is the first Chief secretary of Nepal ?
  \begin{choices}
  \choice Hari Prasad Pradhan
  \choice Anerudra Prasad Singh
  \CorrectChoice Chandra Bahadur Thapa
  \choice Bhes Bahadur Thapa
  \end{choices}

\question Which currency is also known as paper gold ?
  \begin{choices}
  \choice American dollar
  \choice Sterling pound
  \CorrectChoice SDR
  \choice Euro
  \end{choices}

\question Which countries in order approved the Nepal's proposition of being declared peace zone ?
  \begin{choices}
  \CorrectChoice China, Pakistan, North Korea, South Korea, Bangladesh
  \choice China, North Korea, South Korea, Pakistan, Bangladesh
  \choice China, Pakistan, South Korea, North Korea, Bangladesh
  \choice China, Pakistan, Bangladesh, North Korea, South Korea
  \end{choices}

\question Who is the fist elected chief minster of Nepal (provincial)
  \begin{choices}
  \choice Sherdhan Rai
  \choice Lalbabu Raut
  \choice Sankar Pokhrel
  \CorrectChoice Dormani Poudel
  \end{choices}

\question What is the height of TIA above sea level ?
  \begin{choices}
  \choice 1355
  \choice 1350
  \CorrectChoice 1337
  \choice 1400
  \end{choices}

\question When did the Betrawati treaty took place among Nepal, China and Tibet ?
  \begin{choices}
  \choice 1843 BS
  \choice 1845 BS
  \choice 1846 BS
  \CorrectChoice 1849 BS
  \end{choices}

\question Which of the following sayings is incorrect ?
  \begin{choices}
  \choice Nepal's major food crop is Rice
  \choice Biratnagar Jute mill is the first industry ever established in Nepal
  \CorrectChoice Nepal is WTO's 157th member
  \choice World population day is celebrated in July 11, every year
  \end{choices}

\question Which of the following caste does not come under marginalized group ?
  \begin{choices}
  \choice Sunuwar
  \choice Tamang
  \choice Dura
  \CorrectChoice Chepang
  \end{choices}

\question Considering the points below, select which applies.
  \begin{enumerate}
  \item Nepal's easternmost point lies in Taplejung district
  \item Nepal's westernmost point lies in Dodhara of Kanchanpur district
  \item Nepal's northernmost point lies in Changla of Humla district
  \item Nepal's southernmost point lies in Lodhabari of Jhapa district
  \end{enumerate}

  \begin{choices}
  \choice 1, 2 and 3 are correct
  \choice 1, 3 and 4 are correct
  \choice 1, 4 and 2 are correct
  \CorrectChoice All are correct
  \end{choices}

\question Which of the following matches is incorrect ?
  \begin{choices}
  \CorrectChoice Nepal's first king female embassador is Bindeshwori Shah
  \choice Nepal was represented in Sugauli Treaty by Gajaraj Mishra
  \choice It was decided to state Government of Nepal instead of His Majesty's Government in 15 Baisakh, 2063
  \choice Constitutional assembly election 2 was conducted in Mangsir 4, 2070
  \end{choices}

\question Bull fighting is the National game of:
  \begin{choices}
  \choice Italy
  \choice Poland
  \CorrectChoice Spain
  \choice Finland
  \end{choices}

\question Which country lost in Balkan War ?
  \begin{choices}
  \CorrectChoice Turkey
  \choice Serbia
  \choice Greece
  \choice Albania
  \end{choices}

\question Which of the following sayings is incorrect ?
  \begin{choices}
  \choice The book that enlists endangered species worldwide is called Red Book.
  \choice The Red Book came into practice since 1966 AD
  \CorrectChoice Currently Nepal is implementing Climate Change Policy, 2072
  \choice Greenhouse effect was discovered in 1926 AD
  \end{choices}

\question Which of the following match is incorrect ?
  \begin{choices}
  \choice World Environment Day $\longrightarrow$ June 5
  \choice World Ocean Day $\longrightarrow$ June 8
  \choice World Day against Child Labor $\longrightarrow$ June 14
  \CorrectChoice World Refugee Day $\longrightarrow$ June 22
  \end{choices}

\question Which among the following is not related to Industrial sector ?
  \begin{choices}
  \choice Dairy Development Corporation
  \choice Hetauda Cement Industries
  \choice Nepal Medicine Limited
  \CorrectChoice Agriculture Input Company Limited
  \end{choices}

\question SAARC summit and venues are given below. Identify the mismatch.
  \begin{choices}
  \CorrectChoice 14th $\longrightarrow$ Dhaka
  \choice 15 $\longrightarrow$ Colombo
  \choice 16 $\longrightarrow$ Thimpu
  \choice 17 $\longrightarrow$ Addau Sahar
  \end{choices}

\question Who among the following personalities were jailed while being declared recipient of the Nobel prize ?
  \begin{choices}
  \choice Carl von Ossietzky
  \choice Aung San Suu Kyi
  \choice Liu Xiaobo
  \CorrectChoice All of above
  \end{choices}

\question Since when did public transport system started in Nepal ?
  \begin{choices}
  \choice 1994 BS
  \choice 1996 BS
  \choice 1998 BS
  \choice 1999 BS
  \end{choices}

\question Consider the following and select all that apply.
  \begin{enumerate}
  \item Madam Curie received the 1903 Nobel prize in Physics
  \item Madam Curie shared the prize with Henri Becquerel for discovery fo Radioactivity.
  \end{enumerate}

  \begin{choices}
  \choice 1 is correct
  \choice 2 is correct
  \CorrectChoice Both are correct
  \choice None are correct
  \end{choices}

\question Which among following ASEAN nations is landlocked ?
  \begin{choices}
  \choice Malasiya
  \choice Philippines
  \CorrectChoice Laos
  \choice Brunei
  \end{choices}

\question Study and distinguish among following.
  \begin{enumerate}
  \item Amazon river originates from Andes range
  \item Volga is the biggest river of Europe
  \item Nile, the longest river of Africa, flows only in Egypt
  \item Mississipi-Missouri is the largest river of North America
  \end{enumerate}
  \begin{choices}
  \choice 1 and 2 are correct
  \choice 1, 2 and 3 are correct
  \choice All are correct
  \CorrectChoice Only 3 is correct
  \end{choices}

\question Which among the following planets takes the longest in revolving around the sun ?
  \begin{choices}
  \choice Jupiter
  \choice Saturn
  \choice Uranus
  \CorrectChoice Neptune
  \end{choices}

\question Based on geography Asia is divided into:
  \begin{choices}
  \choice 9 parts
  \choice 7 parts
  \choice 6 parts
  \CorrectChoice 5 parts
  \end{choices}

\question Amargadhi is in:
  \begin{choices}
  \choice Darchula district
  \choice Bajhang district
  \choice Bajura district
  \CorrectChoice Dadeldhura district
  \end{choices}

\question Louis-XVI of france was sentenced for death in:
  \begin{choices}
  \choice 1789 AD
  \choice 1791 AD
  \CorrectChoice 1793 AD
  \choice 1795 AD
  \end{choices}

\question \fillin[][2cm] used to only feed himself after all ate.
  \begin{choices}
  \choice Pratap Malla
  \CorrectChoice Mahendra Malla
  \choice Sivasingha Malla
  \choice Ananta Malla
  \end{choices}

\question "Chyabung" dance is a popular dance of:
  \begin{choices}
  \CorrectChoice Limbu
  \choice Sherpa
  \choice Lepcha
  \choice Dhimal
  \end{choices}

\question Who was the prime minister of Nepal during second world war ?
  \begin{choices}
  \choice Bir Samsher
  \choice Juddha Samsher
  \CorrectChoice Chandra Samsher
  \choice Dev Samsher
  \end{choices}

\question Ministry of Tourism was established in:
  \begin{choices}
  \choice 2032
  \CorrectChoice 2033
  \choice 2034
  \choice 2035
  \end{choices}

\question Bado tribes of Nepal are known by name:
  \begin{choices}
  \choice Tharu
  \choice Chepang
  \CorrectChoice Meche
  \choice Rautey
  \end{choices}

\question Which instrument is used for measuring air pressure ?
  \begin{choices}
  \CorrectChoice Manometer
  \choice Ampere
  \choice Bushels
  \choice Lactometer
  \end{choices}

\question A-1 satellite was launched by:
  \begin{choices}
  \CorrectChoice France
  \choice Germany
  \choice Japan
  \choice Britain
  \end{choices}

\question Which language got recognized the latest in the UN
  \begin{choices}
  \choice English
  \CorrectChoice Arabian
  \choice French
  \choice Russian
  \end{choices}

\question The diameter of the Earth is:
  \begin{choices}
  \choice 40800 km
  \choice 40600 km
  \choice 40400 km
  \choice 40200 km
  \end{choices}

\question Which is the biggest volcanic mountain of below:
  \begin{choices}
  \choice Himalayan range
  \choice Mount K2
  \choice Mainaloba
  \choice Ural
  \end{choices}

\question Based on area, which series represents the smallest to largest countries ?
  \begin{choices}
  \choice Russia, Canada, India, Australia, Brasil, China
  \choice Russia, Canada, China, India, Australia, Brasil
  \choice Russia, Canada, China, Brasil, Australia, India
  \choice Russia, Canada, China, Australia, India, Brasil
  \end{choices}

\question Nepal's border is adjoining with \fillin[][2cm] Indian states.
  \begin{choices}
  \choice 4
  \choice 5
  \choice 6
  \choice 7
  \end{choices}

\question Tilicho lake is \fillin[][2cm] masl.
  \begin{choices}
  \choice 3750
  \choice 4319
  \choice 4650
  \choice 4919
  \end{choices}

\question There are \fillin[][2cm] species of turtoise in Nepal.
  \begin{choices}
  \choice 12
  \choice 14
  \choice 10
  \choice 16
  \end{choices}

\question There were \fillin[][2cm] states in USA when it became soverign.
  \begin{choices}
  \choice 9
  \choice 11
  \choice 13
  \choice 15
  \end{choices}

\question Pashupatinath temple was conceived by:
  \begin{choices}
  \choice Prachandadev
  \choice Yognarendra Malla
  \choice Shankardev
  \choice Nyayadev
  \end{choices}

\question Which tribes believes that it is sinful to touch money ?
  \begin{choices}
  \choice Kusunda
  \choice Thami
  \choice Raute
  \choice Chepang
  \end{choices}

\question BBC started broadcast in Nepali since:
  \begin{choices}
  \choice 1967 June 7
  \choice 1967 March 7
  \choice 1967 July 7
  \choice 1967 April 7
  \end{choices}

\question Who was the king of Gorkha before Drabya Shah conquered ?
  \begin{choices}
  \choice Karki
  \choice Bogati
  \choice Khadka magar
  \choice Thapa magar
  \end{choices}

\question Which country has the highest literacy rate among the SAARC countries ?
  \begin{choices}
  \choice India
  \choice Afganistan
  \choice Maldives
  \choice Bangladesh
  \end{choices}

\question What is the word count of Nepalese national anthem.
  \begin{choices}
  \choice 26
  \choice 36
  \choice 46
  \choice 56
  \end{choices}

\question Mahendra cave is in:
  \begin{choices}
  \choice Kaski
  \choice Pyuthan
  \choice Taplejung
  \choice Khotang
  \end{choices}

\question Which is the nearest mountain to Kathmandu ?
  \begin{choices}
  \choice Gaurishankar
  \choice Ganesh
  \choice Machhapuchre
  \choice Jugal
  \end{choices}

\question Consider following statements and select which is true.
  \begin{choices}
  \choice When France was undergoing national revolution Louis XVI was the king.
  \choice When France was undergoing national revolution Marie Antoniete was the queen.
  \choice Both king and queen were guillotened for death sentence in 1994
  \choice Voltaire argued that corrupted churches should be destroyed.
  \end{choices}

\question The last king of Bhaktapur was:
  \begin{choices}
  \choice Bhupatindra Malla
  \choice Pratap Malla
  \choice Jayasthiti Malla
  \choice Ranajit Malla
  \end{choices}

\question Who led the battle of Nalapani ?
  \begin{choices}
  \choice Bhimsen Thapa
  \choice Kalu Pandey
  \choice Balbhadra Kunwar
  \choice Gagasingh
  \end{choices}

\question Lamosanghu-Jiri road section was constructed in assistance of:
  \begin{choices}
  \choice Japan
  \choice Switzerland
  \choice Germany
  \choice Britain
  \end{choices}

\question Who discovered Saturn ?
  \begin{choices}
  \choice Albert Einstein
  \choice Gallileo
  \choice Yuri Gagrin
  \choice Archimedes
  \end{choices}

\question Mention if the given statement is TRUE or FALSE.
  \begin{parts}
  \part There are 5 soil types found in Nepal. \hfill (T/F)
  \part Red silty type soil is suitable for Fingermillet cultivation. \hfill (T/F)
  \part Nepal initiated organized approach for soil conservation since 2002. \hfill (T/F)
  \part Afforestation was practiced in Nepal for the first time in 2002 BS. \hfill (T/F)
  \end{parts}

\end{questions}


\section*{\fullwidth{\Large \centering \textbf{Multiple choice: Accounting}}}
\begin{questions}

\question Generally what should be the ratio of internal rate of return and cost of capital rate ?
  \begin{choices}
  \choice Less than cost of capital rate
  \choice More than cost of capital rate
  \choice Equal of cost of capital rate
  \choice None of the above
  \end{choices}

\question Identify the importance of capital budgeting form the list given below:
  \begin{choices}
  \choice Long lasting impact
  \choice Involves substantial amount of funds
  \choice Decision is not flexible
  \choice Ignore current assets
  \end{choices}

\question What is called a relation between earning before tax and earning before interest and tax ?
  \begin{choices}
  \choice Financial leverages
  \choice Combined leverages
  \choice Operating leverages
  \choice None of the above
  \end{choices}

\question Financial statements are also known as ... statements.
  \begin{choices}
  \choice Behavioral
  \choice Historical
  \choice Research
  \choice Preliminary
  \end{choices}

\question Which of the following is correct relating to basic rules of Double entry accounting ?
  \begin{choices}
  \choice Should write debit on the right and credit on the left
  \choice For every amount of debit there must be same amount of credit
  \choice Debit and credit amount need not be equal
  \choice None of the above
  \end{choices}

\question Which act constitutes Accounting Standard Board:
  \begin{choices}
  \choice Fiscal act
  \choice Audit act
  \choice Appropriation act
  \choice Nepal chartered accountant institutue act
  \end{choices}

\question What status shows in balance sheet of company:
  \begin{choices}
  \choice Status of market management
  \choice Status of profit and loss
  \choice Status of capital and assets
  \choice Status of cost of production
  \end{choices}

\question Which amount does not consist of financial statements of company:
  \begin{choices}
  \choice Balance sheet
  \choice Profit and loss account
  \choice Cost account
  \choice None of the above
  \end{choices}

\question Goodwill is:
  \begin{choices}
  \choice Fictious asset
  \choice Tangible asset
  \choice Intangible asset
  \choice Current asset
  \end{choices}

\question Who can liquidate a company:
  \begin{choices}
  \choice Law
  \choice Person
  \choice Government
  \choice None of the above
  \end{choices}

\question Identify true or false in following statements
  \begin{enumerate}
  \item The main objective of the current company act is to make the incorporation operation and administration of companies much easeier, simpler and more transparent.
  \item According to company act, 2063 company means an organization incorporated under this act.
  \end{enumerate}
  \begin{choices}
  \choice 1 is true but 2 is false
  \choice Both statements are true
  \choice Both statements are false
  \choice 1 is false but 2 is true
  \end{choices}

\question Identify True or False in following statements:
  \begin{enumerate}
  \item Every company should maintain its accounts according to the double entry accounting system.
  \item Every company should maintain its accounts according to the single entry accounting system.
  \end{enumerate}

  \begin{choices}
  \choice 1 is true but 2 is false
  \choice 1 is false but 2 is true
  \choice Both statements are true
  \choice Both statements are false
  \end{choices}

\question Reason for business failure does not include the following aspects:
  \begin{choices}
  \choice Lack of sufficient fund
  \choice Frequent change in governemnt policies
  \choice Lack of quality products
  \choice Managerial efficiency
  \end{choices}

\question Identify True or False in following statements:
  \begin{enumerate}
  \item The general meeting of a company shall be annual and biannual.
  \item The general meeting of a company shall be annual and the extra ordinary.
  \end{enumerate}
  \begin{choices}
  \choice 1 is true but 2 is false
  \choice 1 is false but 2 is true
  \choice Both statements are true
  \choice Both statements are false
  \end{choices}

\question According to Nepal public sector accounting system, which of the following defines cash basis ?
  \begin{choices}
  \choice Cash flows
  \choice Cash receipts
  \choice Cash payments
  \choice A basis of accounting that recognizes transactions and other events only when cash is received or paid
  \end{choices}

\question Which of the following defines subsidiary company (According to the company act)?
  \begin{choices}
  \choice Private company
  \choice Public company
  \choice A company owned by a holding company
  \choice Listed company
  \end{choices}

\question Identify True or False in following statements:
  \begin{enumerate}
  \item The audit committee of a company should consist of at least three members.
  \item Chapter 10 of the current company act described about voluntary liquidation of a company
  \end{enumerate}
  \begin{choices}
  \choice 1 is true but 2 is false
  \choice 1 is false but 2 is true
  \choice Both statements are true
  \choice Both statements are false
  \end{choices}

\question The accounting system provides ... and ... financial information to interested parties.
  \begin{choices}
  \choice Relevant and reliable
  \choice Efficient and effective
  \choice Inclusive and equitable
  \choice Praticipative and responsible
  \end{choices}

\question Identify true or false in following statements:
  \begin{enumerate}
  \item Company should maintain its accounts in the English or the Nepali language.
  \item Company should maintain its accounts only in the Nepali language.
  \end{enumerate}
  \begin{choices}
  \choice 1 is true but 2 is false
  \choice 1 is false but 2 is true
  \choice Both statements are true
  \choice Both statements are false
  \end{choices}

\question Match the contents of the following and select the correct answer:
  \begin{table}[h]
  \centering
  \begin{tabular}{lll}
    a. Financial accounting & & 1. Receipt and payment account \\[2mm]
    b. Cost accounting & & 2. ABC analysis \\
    c. Management accounting & & 3. BEP analysis \\
    d. Accounting for professional & & 4. Profit and loss account \\
  \end{tabular}
  \end{table}
  \begin{choices}
  \choice a-4, b-2, c-3, d-1
  \choice a-1, b-4, c-2, d-3
  \choice a-1, b-3, c-4, d-2
  \choice a-2, b-3, c-1, d-4
  \end{choices}

\question On the basis of following information determine intrinsic value per share and select correct answer:
  \begin{table}[h]
  \centering
  \begin{tabular}{lll}
    Description & Vendor company & Purchasing company \\[2mm]
    Total assets (NRs) & 500000 & 12000000 \\
    Trade liabilities (NRs) & 170000 & 300000 \\
    Share value each costing Rs 100 & 150000 & 300000 \\
  \end{tabular}
  \end{table}
  (Information: The holder of every three shares in vendor company were to receive two shares in purchasing company plus cash as necessary to adjustment.)
  \begin{choices}
  \choice 220 for vendor company and 300 for purchasing company
  \choice 250 for vendor company and 320 for purchasing company
  \choice 300 for vendor company and 300 for purchasing company
  \choice 300 for vendor company and 350 for purchasing company
  \end{choices}

\question In which date prevailing company act comes into force ?
  \begin{choices}
  \choice 2052
  \choice 2053
  \choice 2054
  \choice 2055
  \end{choices}

\question Match the contents of the following and select the correct answer:
  \begin{table}[h]
  \centering
  \begin{tabular}{lll}
    a. Liquidation & & 1. Company operation continues \\[2mm]
    b. Amalagamation & & 2. Jointly new company establishment \\
    c. Absorption & & 3. Internal reconstruction \\
    d. Reconstruction & & 4. Company registrar office \\
  \end{tabular}
  \end{table}
  \begin{choices}
  \choice a-4, b-2, c-1, d-3
  \choice a-1, b-4, c-2, d-3
  \choice a-1, b-3, c-4, d-2
  \choice a-2, b-3, c-1, d-4
  \end{choices}

\question In which of the following conditions company goes into liquidation ?
  \begin{choices}
  \choice By the order of Company registrar
  \choice By the order of court
  \choice By the decision of cabinet
  \choice None of the above
  \end{choices}

\question In which priority the payment of liquidation expenses lies ?
  \begin{choices}
  \choice First
  \choice Second
  \choice Third
  \choice Fourth
  \end{choices}

\question Which of the following account is not a requirement for professional men ?
  \begin{choices}
  \choice Receipt and payment A/C
  \choice Income and expenditure A/C
  \choice Balance sheet
  \choice Profit and loss A/C
  \end{choices}

\question Match the correct pair.
  \begin{table}[h]
  \centering
  \begin{tabular}{lll}
    a. Assets & & 1. Creditors \\[2mm]
    b. Liabilities & & 2. Ratio \\
    c. Equity & & 3. Interest income \\
    d. Revenue & & 4. Divident \\
  \end{tabular}
  \end{table}
  \begin{choices}
  \choice a-2, b-1, c-4, d-3
  \choice a-1, b-4, c-2, d-3
  \choice a-1, b-3, c-4, d-2
  \choice a-2, b-3, c-1, d-4
  \end{choices}

\question Accounting for profession includes all of the following account expect:
  \begin{choices}
  \choice Receipt and payment account
  \choice Income and expenditure account
  \choice Balance sheet
  \choice Government account
  \end{choices}

\question Which one is not the qualitative characteristic of financial statements ?
  \begin{choices}
  \choice Relevance
  \choice Reliability
  \choice Comparability
  \choice Inclusiveness
  \end{choices}

\question Generally accepted accounting principle GAAP includes all the following except:
  \begin{choices}
  \choice Business entity concept
  \choice Going concern concept
  \choice Cost concept
  \choice Inefficiency concept
  \end{choices}

\question The number of shareholders of a private company should not exceed ...
  \begin{choices}
  \choice Fifty
  \choice Sixty
  \choice Seventy
  \choice Eighty
  \end{choices}

\question Except as otherwise, the paid up capital of a public company should be a minimum of ...
  \begin{choices}
  \choice 10 million
  \choice 20 million
  \choice 30 million
  \choice 40 million
  \end{choices}

\question According to current company act, the term officer includes all the following except:
  \begin{choices}
  \choice Director
  \choice Chief executive
  \choice Manager
  \choice Shareholder
  \end{choices}

\question Every public company's Board of Directors should prepare its annual financial statements at least ... days prior to the holding of its annual general meeting.
  \begin{choices}
  \choice 15
  \choice 20
  \choice 25
  \choice 30
  \end{choices}

\question Who among following is the internal user of Accounting information ?
  \begin{choices}
  \choice Creditor
  \choice Government
  \choice Personnel
  \choice Bank
  \end{choices}

\question Which among the following is correct ? Generally stock value of the bank increases, when:
  \begin{enumerate}
  \item Profit of the bank increases
  \item In the condition of divident increment
  \item Risk reduction by reducing debt
  \item Increase of capital fund
  \end{enumerate}
  \begin{choices}
  \choice All are correct
  \choice All are incorrect
  \choice a and b are correct
  \choice c and d are correct
  \end{choices}

\question Ratio between bank's revenue and assets is called:
  \begin{choices}
  \choice Net creditor limit
  \choice Net profit margin
  \choice Net operation margin
  \choice Bank performance
  \end{choices}

\question Why trial balance is prepared in a company ?
  \begin{choices}
  \choice To identify profit/loss
  \choice To examine mathematical accuracy of ledger
  \choice To trace the cash balance
  \choice None of the above
  \end{choices}

\question What is amalgamation of companies ?
  \begin{choices}
  \choice Absorption of one company
  \choice Reorganization of two or more companies
  \choice Reconstruction of a company
  \choice Reformation of a company
  \end{choices}

\question 50\% or more of total capital of a company invested as share investment by another company is called ...
  \begin{choices}
  \choice Subsidiary company
  \choice Holding company
  \choice New company
  \choice None of the above
  \end{choices}

\question AMAN company purchased 8000 shares of BIMAN company at the rate of Rs 150 per share. Actual share value of BIMAN company is only Rs 100 per share. Capital gain of Rs 150000 is allotted to AMAN company. On the base of these information calculate the value of Goodwill.
  \begin{choices}
  \choice Rs 200000
  \choice Rs 250000
  \choice Rs 300000
  \choice Rs 400000
  \end{choices}

\question In which of the following condition name of the company need not be changed ?
  \begin{choices}
  \choice Amalgamation
  \choice Absorption
  \choice Reconstruction
  \choice Internal reconstruction
  \end{choices}

\question Which of the following transaction is not included in Consolidated Balance sheet ?
  \begin{choices}
  \choice Profit before take-over
  \choice Profit after take-over
  \choice Loss after take-over
  \choice Inter company transaction
  \end{choices}

\question Nath Co. Ltd sells its business to Anath Co. Ltd. The balance sheet of Nath Co. Ltd on the data is as follows:
  \begin{table}[h]
  \centering
  \begin{tabular}{llll}
    \textbf{Liabilities} & & \textbf{Assets} & \\[2mm]
    Issued capital & 150000 & Plant and machinery & 180000 \\
    Creditor & 100000 & Furniture & 30000 \\
    Workmen fund & 30000 & Cash & 30000 \\
    Debenture & 50000 & Debtors & 100000 \\
    Profit/loss & 15000 & Preliminary expenses & 5000 \\
    Total & 345000 & Total & 345000
  \end{tabular}
  \end{table}
  Which is correct purchase consideration under net asset method:
  \begin{choices}
  \choice 220000
  \choice 230000
  \choice 240000
  \choice 250000
  \end{choices}

\question Identify True or False in following statements:
  \begin{enumerate}
  \item Net income = Total revenue - Total expenditure
  \item Net income = Total revenue - Total debt
  \end{enumerate}
  \begin{choices}
  \choice 1 is true but 2 is false
  \choice Both statements are true
  \choice Both statements are false
  \choice 1 is false but 2 is true
  \end{choices}

\question Match the following:
  \begin{table}[h]
  \centering
  \begin{tabular}{lll}
    Setting account standard & Auditing standard board \\[2mm]
    Setting auditing standard & Accounting standard board \\
    Developing of accounting profession & Auditor general \\
    Development of accounting profession & Institute of charter accounting of nepal \\
  \end{tabular}
  \end{table}
  \begin{choices}
  \choice a-3, b-4, c-2, d-1
  \choice a-2, b-1, c-4, d-3
  \choice a-4, b-3, c-2, d-1
  \choice a-1, b-2, c-3, d-4
  \end{choices}

\question Identify the most obvious users of accounting information person is also called ...
  \begin{choices}
  \choice Government and its agencies
  \choice Investors and lenders
  \choice Development partners
  \choice Political parties
  \end{choices}

\question Accounting equation prepared by professional person is called ...
  \begin{choices}
  \choice Capital equation
  \choice Balance sheet equation
  \choice Profit and loss
  \choice Liability equation
  \end{choices}

\question As per income tax act which one of following is not professional person ?
  \begin{choices}
  \choice Doctor
  \choice Auditor
  \choice Lawyer
  \choice Farmer
  \end{choices}

\question Read the following statements and identify the correct and incorrect alternative:
  \begin{enumerate}
  \item Capital income and expenditures are not included in income and expenditure account
  \item Balance sheet is position statement of business
  \item Income and expenditure account based on real account
  \end{enumerate}
  \begin{choices}
  \choice 1 and 2 are correct but 3 is incorrect
  \choice 1 is correct but 2 and 3 are incorrect
  \choice 2 is correct but 1 and 2 are incorrect
  \choice 1 and 2 are correct but 3 is incorrect
  \end{choices}

\question What types of transactions between holding and subsidiary company are not shown in consolidated balance sheet ?
  \begin{choices}
  \choice Profit after holding company
  \choice Loss after holding company
  \choice Profit before holding company
  \choice Inter company transaction
  \end{choices}

\question What is the name of profit before making investment in subsidiary company by holding company ?
  \begin{choices}
  \choice Capital profit
  \choice Revenue profit
  \choice Pre-acquisition profit
  \choice Unrealized profit
  \end{choices}

\question What is the additional amount of money paid by the holding company ?
  \begin{choices}
  \choice Premium
  \choice Interest
  \choice Discount
  \choice Goodwill
  \end{choices}

\question
  \begin{choices}
  \choice
  \choice
  \choice
  \choice
  \end{choices}

\question
  \begin{choices}
  \choice
  \choice
  \choice
  \choice
  \end{choices}

\question
  \begin{choices}
  \choice
  \choice
  \choice
  \choice
  \end{choices}

\question
  \begin{choices}
  \choice
  \choice
  \choice
  \choice
  \end{choices}

\question
  \begin{choices}
  \choice
  \choice
  \choice
  \choice
  \end{choices}

\question
  \begin{choices}
  \choice
  \choice
  \choice
  \choice
  \end{choices}

\question
  \begin{choices}
  \choice
  \choice
  \choice
  \choice
  \end{choices}

\question
  \begin{choices}
  \choice
  \choice
  \choice
  \choice
  \end{choices}

\question
  \begin{choices}
  \choice
  \choice
  \choice
  \choice
  \end{choices}

\question
  \begin{choices}
  \choice
  \choice
  \choice
  \choice
  \end{choices}

\question
  \begin{choices}
  \choice
  \choice
  \choice
  \choice
  \end{choices}

\question
  \begin{choices}
  \choice
  \choice
  \choice
  \choice
  \end{choices}

\question
  \begin{choices}
  \choice
  \choice
  \choice
  \choice
  \end{choices}

\question
  \begin{choices}
  \choice
  \choice
  \choice
  \choice
  \end{choices}

\question
  \begin{choices}
  \choice
  \choice
  \choice
  \choice
  \end{choices}

\question
  \begin{choices}
  \choice
  \choice
  \choice
  \choice
  \end{choices}

\question
  \begin{choices}
  \choice
  \choice
  \choice
  \choice
  \end{choices}

\question
  \begin{choices}
  \choice
  \choice
  \choice
  \choice
  \end{choices}

\question
  \begin{choices}
  \choice
  \choice
  \choice
  \choice
  \end{choices}

\question
  \begin{choices}
  \choice
  \choice
  \choice
  \choice
  \end{choices}

\question
  \begin{choices}
  \choice
  \choice
  \choice
  \choice
  \end{choices}

\question
  \begin{choices}
  \choice
  \choice
  \choice
  \choice
  \end{choices}

\question
  \begin{choices}
  \choice
  \choice
  \choice
  \choice
  \end{choices}

\question
  \begin{choices}
  \choice
  \choice
  \choice
  \choice
  \end{choices}

\question
  \begin{choices}
  \choice
  \choice
  \choice
  \choice
  \end{choices}

\question
  \begin{choices}
  \choice
  \choice
  \choice
  \choice
  \end{choices}



\subsection*{\underline{General subjects}}

\question UNO has declared to celebrate year 2017 as:
  \begin{choices}
  \choice Poverty alleviation
  \choice Year of sustainable tourism for development
  \choice Sustainable development and peace
  \choice Youth for development
  \end{choices}

\question In which chapter ( \textit{Dhara}) of Nepalese constitution, there is statement that \textit{Madhesi}, \textit{Tharu} and \textit{Muslim} comission will be reconsidered after 10 years by assembly house:
  \begin{choices}
  \choice Dhara 264
  \choice Dhara 266
  \choice Dhara 265
  \choice Dhara 267
  \end{choices}

\question Which country was the first to switch off Radio Network ?
  \begin{choices}
  \choice Norway
  \choice US
  \choice Australia
  \choice Denmark
  \end{choices}

\question According to World Economic Forum, 2016, which country is regarded to have most inclusive development ?
  \begin{choices}
  \choice Lithuania
  \choice Sweden
  \choice Mauritiana
  \choice Denmark
  \end{choices}

\question Like NEPSE for Nepal, India has:
  \begin{choices}
  \choice ISE
  \choice INE
  \choice BSE
  \choice INX
  \end{choices}

\question Which among the following are UN peace keeping missions (multiple or none)
  \begin{choices}
  \choice UNMIL
  \choice UNFIL
  \choice UNMIN
  \choice MINUSTAH
  \end{choices}

\question Daniel Ortega has recently become president for third time of:
  \begin{choices}
  \choice Senegal
  \choice Nicaragua
  \choice Lithuania
  \choice Mauritiana
  \end{choices}

\question "The people's president: Dr. APJ Abdul Kalam" was written by:
  \begin{choices}
  \choice Hamid Ansari
  \choice SM Khan
  \choice Meghna Pant
  \choice Akshaya Mukul
  \end{choices}

\question Group of 77 is currently chaired by which country?
  \begin{choices}
  \choice Thailand
  \choice Figi
  \choice Equador
  \choice Algeria
  \end{choices}

\question International tuberculosis day is celebrated in remembrance of:
  \begin{choices}
  \choice Abraham Lincoln
  \choice Mahatma Gandhi
  \choice Nelsen Mandela
  \choice Edward Joner
  \end{choices}

\end{questions}
