\section*{General Knowledge: Objective}

\begin{questions}

\fullwidth{\large \centering \textbf{Multiple choice}}

%% in order to provide indication for marks of each question, do it as follows:
% \question[1] What is the relationship between latitude and temperature ?
\question What is the relationship between latitude and temperature ?
  \begin{enumerate}
  \item With increase in latitude, temperature increases
  \item With increase in latitude, temperature decreases
  \item With decrease in latitude, temperature increases
  \item With decrease in latitude, temperature decreases
  \end{enumerate}

  \begin{choices}
  \choice 1 and 3 are correct
  \choice 1 and 2 are correct
  \CorrectChoice 2 and 3 are correct
  \choice 3 and 4 are correct
  \end{choices}

\question Large and bright meteors are called:
  % this places all choices in separate lines
  \begin{choices}
  \choice Falling star
  \choice Shooting star
  \CorrectChoice Fire ball
  \choice Shooting ball
  \end{choices}

\question The country with most languages spoken is:
  % this places all choices in the same line

  \begin{oneparchoices}
  \choice China
  \CorrectChoice India
  \choice America
  \choice Russia
  \end{oneparchoices}

\question Which is the biggest of the deserts ?

  \begin{oneparchoices}
  \choice Somali
  \CorrectChoice Karakoram
  \choice Thar
  \choice Attacama
  \end{oneparchoices}

\question Popular piligrimage Muktinath occurs in the elevation of:
  \begin{choices}
  \choice 4500 m
  \CorrectChoice 3750 m
  \choice 3500 m
  \choice 4100 m
  \end{choices}

\question Which amongst these nations have been elected in UN security council for most number of times?
  \begin{choices}
  \choice Japan
  \choice Spain
  \choice Nepal
  \CorrectChoice Uruguay
  \end{choices}

\question Chess is the national game of:

  \begin{oneparchoices}
  \choice India
  \CorrectChoice Russia
  \choice Saudi Arabia
  \choice none of above
  \end{oneparchoices}

\question Nepal is not the earliest among south asian nations in:
  \begin{choices}
  \choice Maintaining diplomatic policy with Israel
  \CorrectChoice Incorporating "Rights to information" in constitution
  \choice Formulating a separate law regarding "Rights to information"
  \choice Summiting Mt. Everest.
  \end{choices}

\question Which among the below is known as the city of handcrafts:
  \begin{choices}
  \choice Bhaktapur
  \CorrectChoice Patan
  \choice Bungmati
  \choice Bandipur
  \end{choices}

\question Nepal is the \fillin[][2cm] nation to have issued constitution through a constitutional assembly.
  \begin{choices}
  \choice 28
  \choice 35
  \CorrectChoice 44
  \choice 53
  \end{choices}

\question Which country among the following is the first to grant voting rights to women ?
  \begin{choices}
  \choice America
  \choice Greece
  \CorrectChoice New Zealand
  \choice Japan
  \end{choices}

\question Nepal started elephant polo completion in the year:
  \begin{choices}
  \choice 1962 AD
  \CorrectChoice 1982 AD
  \choice 1992 AD
  \choice 2002 AD
  \end{choices}

\question Which among the following cities of Nepal is the first "Eco" city:
  \begin{choices}
  \choice Pokhara
  \choice Madi
  \choice Dharan
  \CorrectChoice Bharatpur
  \end{choices}

\question Which year did the SAARC university started to operate in
  \begin{choices}
  \choice 2008
  \choice 2009
  \CorrectChoice 2010
  \choice 2011
  \end{choices}

\question In how many countries of the world tigers are found?
  \begin{choices}
  \choice 5
  \choice 9
  \CorrectChoice 13
  \choice 14
  \end{choices}

\question When did EuroCup started ?
  \begin{choices}
  \CorrectChoice 1960
  \choice 1964
  \choice 1968
  \choice 1990
  \end{choices}

\question Which among the following cities does not comprise the Silk road?
  \begin{choices}
  \choice Bukhara, Uzbekistan
  \choice Kabul, Afganistan
  \CorrectChoice Jihadh, Saudi Arabia
  \choice Aleppo, Syria
  \end{choices}

\question ASEAN has how many member and supervisor nations?
  \begin{choices}
  \CorrectChoice 10, 2
  \choice 9, 3
  \choice 10, 4
  \choice 10, 5
  \end{choices}

\question The tallest peak of Chure range - Garba - has elevation of:
  \begin{choices}
  \CorrectChoice 1872 m
  \choice 1730 m
  \choice 1827 m
  \choice 1890 m
  \end{choices}

\question "It is our world" is the slogan of \fillin[][3cm] organization:
  \begin{choices}
  \choice WB
  \choice WHO
  \CorrectChoice UNO
  \choice All of above
  \end{choices}

\question BP Koirala inaugurated the BP highway in:
  \begin{choices}
  \CorrectChoice Shreekhandapur, Kavre
  \choice Dumla, Sindhuli
  \choice Bardibas, Mahottari
  \choice Sindhuligadi, Sindhuli
  \end{choices}

\question Nepal's first civil revolution is that of:
  \begin{choices}
  \CorrectChoice Biratnagar Jute mill
  \choice Jhapa revolution
  \choice Parcha revolution
  \choice None
  \end{choices}

\question BOAO forum for Asia summit organized in April 11, 2018 has elected as president:
  \begin{choices}
  \choice Xi Jin Ping
  \choice Li Khichiang
  \CorrectChoice Wan Ki Moon
  \choice Hu Jin Tao
  \end{choices}

\question Based on the office duration of individuals chairing house of representatives of Nepal which among following order is correct:
  \begin{enumerate}
  \item Ramchandra Poudel
  \item Damanath Dhungana
  \item Taranath Ranabhat
  \end{enumerate}
  \begin{choices}
  \choice 2, 3 and 1
  \CorrectChoice 2, 1 and 3
  \choice 1, 2 and 3
  \choice 3, 2 and 1
  \end{choices}

\question Banke, Bardiya, Kailali and Kanchanpur (a.k.a new nation) were returned to Nepal in:
  \begin{choices}
  \choice 1911 BS
  \choice 1915 BS
  \CorrectChoice 1917 BS
  \choice 1919 BS
  \end{choices}

\question According to village executive committee formation guidelines, in a village with 7 wards how many members form the village executive committe ?
  \begin{choices}
  \choice 13
  \choice 14
  \choice 15
  \choice 16
  \end{choices}

\question In the Human rights declaration of December 10, 1948, there are \fillin[][2cm] articles.
  \begin{choices}
  \choice 20
  \CorrectChoice 30
  \choice 32
  \choice 40
  \end{choices}

\question There are \fillin[][2cm] local bodies in Province 3.
  \begin{choices}
  \choice 116
  \choice 117
  \choice 118
  \CorrectChoice 119
  \end{choices}

\question Hulak Goshwara Adda was established during the regime of:
  \begin{choices}
  \choice Bhim Shamsher
  \CorrectChoice Chandra Shamsher
  \choice Juddha Shamsher
  \choice Dev Shamsher
  \end{choices}

\question "Rastriya Sabha Griha" was established with the assistance of \fillin[][3.5cm] while being designed by:
  \begin{choices}
  \CorrectChoice China, Gangadhar Bhatta
  \choice India, Gangadhar Bhatta
  \choice China, Atmakrishna Shrestha
  \choice India, Atmakrishna Shrestha
  \end{choices}

\question Which among the following statements are righly described below:
  \begin{enumerate}
  \item Chilime hydroelectricity project lies in Rasuwa district
  \item Chameliya hydroelectricity project lies in Darchula district
  \end{enumerate}
  \begin{choices}
  \CorrectChoice Both 1 and 2 are correct
  \choice Both 1 and 2 are wrong
  \choice 1 is correct and 2 is wrong
  \choice 2 is correct and 1 is wrong
  \end{choices}

\question 16th Aaha-Rara gold cup, 2019 finals was played between:
  \begin{choices}
  \CorrectChoice Nepal police club and Three star club
  \choice Three star club and Manang marsyangdi club
  \choice Thribhuwan army club and Nepal police
  \choice Nepal police and Sahara club
  \end{choices}

\question Which country does not have the same name for its country and capital?
  \begin{choices}
  \choice Luxemborg
  \choice San marino
  \choice Djibouti
  \CorrectChoice Sicily
  \end{choices}
  \begin{solution}
      All other are country and the names for its capital respectively with exception of Sicily which is the largest island in Mediterranean sea and one of the 20 regions of Italy.
  \end{solution}

\question Based on the area, the largest to smallest deserts of the world are:
  \begin{choices}
  \CorrectChoice Arabian, Gobi, Kalahari, Patagonia
  \choice Gobi, Arabian, Kalahari, Patagonia
  \choice Gobi, Arabian, Patagonia, Kalahari
  \choice Arabian, Gobi, Patagonia, Kalahari
  \end{choices}


\begin{solution}

Following are some of the deserts of the world. This does not enumerate the deserts by order of their sizes:

\begin{itemize}
    \item Arabian 899,618 sq miles. Spanning almost all of Arabian peninsula
    \item Atacama 600 mile long area rich in nitrate and copper deposits in northern chile
    \item Chihuahuan 139769 sq miles, Arizona and Mexico.
    \item Dasht-e Kavir 500 mile long by 200 mile wise in north-central Iran
    \item Dasht-e Lut 300 mile long by 200 mile wide in south-central Iran
    \item Death Valley 3300 sq miles, in California and Nevada
    \item Eastern (Arabian), Egypt. Between the Nile and Red sea extending south into Sudan
    \item Gibson, 60232 sq miles in the interior of western Australia
\end{itemize}

\end{solution}

\question Which of the following states does not have identity of "Baise rajya"?
  \begin{choices}
  \choice Jahari
  \choice Dullu
  \choice Bilashpur
  \CorrectChoice Khanchi
  \end{choices}

\question Which among the following matches is incorrect ?
  \begin{table}[h]
  \centering
  \begin{tabular}{lll}
    \textbf{Landmark} & & \textbf{Location} \\[2mm]
    1. Laddakh range & & Asia \\
    2. Andes range & & South america \\
    3. Alps range & & North america \\
    4. Karakoram range & & Asia \\
  \end{tabular}
  \end{table}
  \begin{choices}
  \choice 1 and 2 are incorrect
  \choice 2 and 3 are incorrect
  \choice Only 1 is incorrect
  \CorrectChoice Only 3 is incorrect
  \end{choices}

\question Which conference embodied the theme that "Women's rights are human rights"?
  \begin{choices}
  \choice The World Conference on Human Rights held by the United Nations in Vienna, Austria, on 14 to 25 June 1993.
  \choice First World Women Conference, Mexico, 1975
  \choice Second World Women Conference, Copenhegan, 1980
  \choice Fourth World Women Conference, Beijing, 1995
  \end{choices}

\question What is the correct order based on chronology of establishment ?
  \begin{choices}
  \choice Radio Nepal, Nepal Telecommunication Corporation, Gorkhapatra, Nepal Television
  \CorrectChoice Gorkhapatra, Radio Nepal, Nepal Telecommunication Corporation, Nepal Television
  \choice Nepal Telecommunication Corporation, Radio Nepal, Gorkhapatra, Nepal Television
  \choice Gorkhapatra, Nepal Telecommunication Corporation, Radio Nepal, Nepal Television
  \end{choices}

\question "For a living planet" is the motto of:
  \begin{choices}
  \choice IUCN
  \choice UNEP
  \CorrectChoice WWF
  \choice UN
  \end{choices}

\question "An agenda for peace" was presented by:
  \begin{choices}
  \CorrectChoice Butroes Butroes Gholi
  \choice Kolfi Annan
  \choice Wan Ki Moon
  \choice None of above
  \end{choices}

\question When is International day of Non-voilence observed ?
  \begin{choices}
  \CorrectChoice October 2
  \choice October 3
  \choice October 4
  \choice October 5
  \end{choices}

\question Cockpit of europe is:
  \begin{choices}
  \choice France
  \choice Germany
  \choice Serbia
  \CorrectChoice Belgium
  \end{choices}

\question Which of the following countries have not been named after river ?
  \begin{choices}
  \choice Paraguay
  \choice Jambia
  \CorrectChoice Sirya
  \choice Niger
  \end{choices}

\question Which district does not have two submetropolitan city ?
  \begin{choices}
  \choice Sunsari
  \CorrectChoice Rupandehi
  \choice Bara
  \choice Dang
  \end{choices}

\question Patagonia desert is in:
  \begin{choices}
  \choice China and congo
  \choice Chile and Argentina
  \CorrectChoice China and Kazakistan
  \choice Argentina and Brasil
  \end{choices}

\question Which of the following recognitions is also known as Nobel prize on environment
  \begin{choices}
  \choice Global 500
  \CorrectChoice Goldman environment prize
  \choice Galante conservation award
  \choice Green peace prize
  \end{choices}

\question Which among the following articles of Nepalese consitution was the first amendment made ?
  \begin{choices}
  \choice Article 43
  \choice Article 84
  \choice Article 167
  \choice Article 283
  \end{choices}

\question Katuwal lake is in:
  \begin{choices}
  \choice Okhaldhunga
  \choice Chitwan
  \CorrectChoice Kathmandu
  \choice Lamjung
  \end{choices}

\question Nagarkot, a tourism site situated in Bhaktapur, is \fillin[][2cm] masl.
  \begin{choices}
  \choice 2250
  \CorrectChoice 2175
  \choice 2275
  \choice 2165
  \end{choices}

\question Gopal dynasty had the capital in:
  \begin{choices}
  \CorrectChoice Gokarna
  \choice Matatirtha
  \choice Godabari
  \choice Bouddha
  \end{choices}

\question Folkland islands were the cause of battle in:
  \begin{choices}
  \CorrectChoice 1981
  \choice 1983
  \choice 1982
  \choice 1984
  \end{choices}

\question Which article of the constitution of Nepal contains provision of constitutional amendment ?
  \begin{choices}
  \CorrectChoice Article 274
  \choice Article 243
  \choice Article 285
  \choice Article 266
  \end{choices}

\question "Point of program" is associated with:
  \begin{choices}
  \choice Road construction
  \choice Agricultural production management
  \choice Organizational restructuring
  \CorrectChoice Foreign aid
  \end{choices}

\question "Village return" campaign was implemented in:
  \begin{choices}
  \choice 2020
  \choice 2021
  \choice 2022
  \CorrectChoice 2024
  \end{choices}

\question Jhimrukh hyrdoelectricity project is situated in:
  \begin{choices}
  \CorrectChoice Pyuthan
  \choice Salyan
  \choice Dailekh
  \choice Syangja
  \end{choices}

\question What is the size of provincial assembly in Province 7 ?
  \begin{choices}
  \CorrectChoice 53 (32 directly elected, 21 proportional)
  \choice 54 (32 directly elected, 22 proportional)
  \choice 53 (34 directly elected, 19 proportional)
  \choice 53 (30 directly elected, 34 proportional)
  \end{choices}

\question The provision regarding president and vice president that each of both should be of different geneder and of a different community is installed in constitution of Nepal's article:
  \begin{choices}
  \choice 70
  \choice 71
  \choice 72
  \CorrectChoice 73
  \end{choices}

\end{questions}

\begin{questions}

\fullwidth{\large \centering \textbf{True or False}}

\question Mention if the given statement is TRUE or FALSE.
  \begin{parts}
  \part There are 5 soil types found in Nepal. \hfill (T/F)
  \part Red silty type soil is suitable for Fingermillet cultivation. \hfill (T/F)
  \part Nepal initiated organized approach for soil conservation since 2002. \hfill (T/F)
  \part Afforestation was practiced in Nepal for the first time in 2002 BS. \hfill (T/F)
  \end{parts}

\end{questions}
