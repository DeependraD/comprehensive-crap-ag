\section*{\fullwidth{\large \centering \textbf{General Knowledge: Quick facts}}}

\subsection*{\fullwidth{\Large \centering \textbf{Sports}}}

\begin{questions}

\question \textbf{Individual records for World cup, 2018}

\begin{solution}

Most goals scored by an individual: 6; Harry Kane

Most assists provided by an individual: 2

Éver Banega, Nacer Chadli, Viktor Claesson, Philippe Coutinho, Kevin De Bruyne, Artem Dzyuba, Aleksandr Golovin, Antoine Griezmann, Eden Hazard, Lucas Hernandez, Lionel Messi, Thomas Meunier, Juan Fernando Quintero, James Rodríguez, Carlos Andrés Sánchez, Youri Tielemans, Wahbi Khazri

Most goals and assists produced by an individual: 6

Antoine Griezmann (4 goals, 2 assists), Harry Kane (6 goals)

Most clean sheets achieved by a goalkeeper: 3

Alisson, Thibaut Courtois, Hugo Lloris, Fernando Muslera, Robin Olsen

Most consecutive clean sheets achieved by a goalkeeper: 3

Alisson, Fernando Muslera

Most goals scored by one player in a match: 3

Harry Kane for England against Panama, Cristiano Ronaldo for Portugal against Spain

Oldest goal scorer: 37 years, 120 days

Felipe Baloy for Panama against England

Youngest goal scorer: 19 years, 183 days

Kylian Mbappé for France against Peru

\end{solution}

\question \textbf{Win loss records for World cup, 2018}

\begin{solution}

Most wins: 6 – Belgium, France

Fewest wins: 0 – Australia, Costa Rica, Egypt, Iceland, Morocco, Panama

Most losses: 3 – Egypt, England, Panama

Fewest losses: 0 – Denmark, France, Spain

Most draws: 3 – Denmark, Spain

Fewest draws: 0 – Belgium, Egypt, Germany, Mexico, Nigeria, Panama, Peru, Poland, Saudi Arabia, Serbia, South Korea, Sweden, Tunisia, Uruguay

Most points in the group stage: 9 – Belgium, Croatia, Uruguay

Fewest points in the group stage: 0 – Egypt, Panama

\end{solution}
\end{questions}

\begin{questions}

\subsection*{\fullwidth{\large \centering \textbf{United Nations and International Organizations}}
}

\question World conference on Women, 1975

\begin{solution}

The conference was held between 19 June and 2 July 1975 in Mexico City, Mexico. It was the first international conference held by the United Nations to focus solely on women's issues and marked a turning point in policy directives. After this meeting, women were viewed as part of the process to develop and implement policy, rather than recipients of assistance. The conference was one of the events established for International Women's Year and led to the creation of both the United Nations Decade for Women and follow-up conferences to evaluate the progress that had been made in eliminating discrimination against women and their equality. Two documents were adopted from the conference proceedings, the World Plan of Action which had specific targets for nations to implement for women's improvement and the Declaration of Mexico on the Equality of Women and Their Contribution to Development and Peace, which discussed how nations foreign policy actions impacted women. It also led to the establishment of the International Research and Training Institute for the Advancement of Women to track improvements and continuing issues and the United Nations Development Fund for Women to provide funding for developmental programs. The conference marked the first time that the parallel Tribune meeting was successful in submitting input to the official meeting and became a catalyst for women's groups to form throughout the world.

\end{solution}

\question United Nations Commission on Status of Women

\begin{solution}

The Commission on the Status of Women (CSW or UNCSW) is a functional commission of the United Nations Economic and Social Council (ECOSOC), one of the main UN organs within the United Nations. CSW has been described as the UN organ promoting gender equality and the empowerment of women. Every year, representatives of Member States gather at United Nations Headquarters in New York to evaluate progress on gender equality, identify challenges, set global standards and formulate concrete policies to promote gender equality and advancement of women worldwide. In April 2017, ECOSOC elected 13 new members to CSW for a four-year term 2018–2022. One of the new members is Saudi Arabia, which has been criticised for its treatment of women.

The commission was formed on 21 June 1946 and is headquartered in NY, USA.

\end{solution}

\end{questions}
