\section*{\fullwidth{\large \centering \textbf{General Knowledge: Nepal Factsheet}}}

\subsection*{\fullwidth{\Large \centering \textbf{Constitution}}}

\begin{questions}

\question What are the major contents of Constitution of Nepal, 2072
  \begin{solution}
  \begin{itemize}
  \item Preamble
  \item Part-1 Preliminary
  \item Part-2 Citizenship
  \item Part-3 Fundamental Rights and Duties
  \item Part-4 Directive Principles, Policies and Obligations of the State
  \item Part-5 Structure of State and Distribution of State Power
  \item Part-6 President and Vice-President
  \item Part-7 Federal Executive
  \item Part-8 Federal Legislature
  \item Part-9 Federal Legislative Procedures
  \item Part-10 Federal Financial Procedures
  \item Part-11 Judiciary
  \item Part- 12 Attorney General
  \item Part -13 State Executive
  \item Part-14 State Legislature
  \item Part-15 State Legislative Procedures
  \item Part -16 State Financial Procedures
  \item Part-17 Local Executive
  \item Part-18 Local Legislature
  \item Part-19 Local Financial Procedures
  \item Part-20 Interrelations between Federation, State and Local Level
  \item Part-21 Commission for the Investigation of Abuse of Authority
  \item Part-22 Auditor General
  \item Part -23 Public Service Commission
  \item Part-24 Election Commission
  \item Part-25 National Human Rights Commission
  \item Part-26 National Natural Resources and Fiscal Commission
  \item Part- 27 Other Commissions
  \item Part-28 Provisions Relating to National Security
  \item Part-29 Provisions Relating to Political Parties
  \item Part-30 Emergency Power
  \item Part-31 Amendment to the Constitution
  \item Part-32 Miscellaneous
  \item Part-33 Transitional Provisions
  \item Part-34 Definitions and Interpretations
  \item Part-35 Short Title, Commencement and Repeal
  \item Schedule-1 National Flag of Nepal
  \item Schedule-2 National Anthem of Nepal
  \item Schedule-3 Coat of Arms of Nepal
  \item Schedule-4 States, and Districts to be included in the concerned States.
  \item Schedule-5 List of Federal Power
  \item Schedule-6 List of State Power
  \item Schedule-7 List of Concurrent Powers of Federation and State
  \item Schedule-8 List of Local Level Power
  \item Schedule-9 List of Concurrent Power of Federation, State and Local level
  \end{itemize}
  \end{solution}

\question What does Part-3 of Consitution of Nepal, 2072 -- Fundamental rights and duties, include ?
  \begin{solution}

  Fundamental rights and duties start at article 16.

  \begin{enumerate}
  \addtocounter{enumiii}{16} % for nested enumerations use enumii, enumiii and enumiv
      \item Right to live with dignity:
      \begin{enumerate}
          \item Every person shall have the right to live with dignity.
          \item No law shall be made providing for the death penalty to any one.
      \end{enumerate}
      \item Right to freedom:
      \begin{enumerate}
          \item No person shall be deprived of his or her personal liberty except in accordance with law.
          \item Every citizen shall have the following freedoms:
          \begin{enumerate}
              \item freedom of opinion and expression,
              \item freedom to assemble peaceably and without arms,
              \item freedom to form political parties,
              \item freedom to form unions and associations,
              \item freedom to move and reside in any part of Nepal,
              \item freedom to practice any profession, carry on any occupation, and establish and operate any industry, trade and business in any part of Nepal.
          \end{enumerate}
      \end{enumerate}
      \item[] Provided that:
      \begin{enumerate}
          \item Nothing in sub-clause (a) shall be deemed to prevent the making of an Act to impose reasonable restrictions on any act which may undermine the sovereignty, territorial integrity, nationality and independence of Nepal or the harmonious relations between the Federal Units or the people of various castes, tribes, religions or communities or incite castebased discrimination or untouchability or on any act of disrespect of labour, defamation, contempt of court, incitement to an offence or on any act which may be contrary to public decency or morality.
          \item Nothing in sub-clause (b) shall be deemed to prevent the making of an Act to impose reasonable restrictions on any act which may undermine the sovereignty, territorial integrity, nationality and independence of Nepal or the harmonious relations between the Federal Units or public peace and order.
          \item Nothing in sub-clause (c) shall be deemed to prevent the making of an Act to impose reasonable restrictions on any act which may undermine the sovereignty, territorial integrity, nationality and independence of Nepal, constitute an espionage against the nation or divulge national secrecy or on any act of rendering assistance to any foreign state, organization or representative in a manner to undermine the security of Nepal or on an act of sedition or on any act which may undermine the harmonious relations between the Federal Units or on any act of incitement to caste-based or communal hatred or on any act which may undermine the harmonious relations between various castes, tribes, religions and communities, or on any act of acquisition of, or restriction on, membership of any political party on the basis solely of tribe, language, religion, community or sex or on any act of formation of a political party with discrimination between citizens or on incitement to violent acts or on any act which may be contrary to public morality. (4) Nothing in sub-clause (d) shall be deemed to prevent the making of an Act to impose reasonable restrictions on any act which may undermine the sovereignty, territorial integrity, nationality and independence of Nepal, or on any act which may constitute espionage against the nation or on any act of divulgence of national secrecy or on any act assisting any foreign state, organization or representative in a manner to undermine the security of Nepal or on an act of sedition or on any act which may undermine the harmonious relations between the Federal Units or on any act of incitement to caste-based or communal hatred or on any act which may undermine the harmonious relations between various castes, tribes, religions and communities or on incitement to violent acts or on any act which may be contrary to public morality.
          \item Nothing in sub-clause (e) shall be deemed to prevent the making of an Act to impose reasonable restrictions on any act which may undermine the interest of the general public or which may undermine the harmonious relations between the Federal Units or the harmonious relations between the peoples of various castes, tribes, religions or communities or which may constitute or incite violent acts.
          \item Nothing in sub-clause (f) shall be deemed to prevent the making of an Act to prevent any act which may undermine the harmonious relations between the Federal Units or any act which may be contrary to public health, decency or morality of the general public or to confer on the State the exclusive right to undertake any specific industry, trade or service, or to prescribe any condition or qualification for carrying on any industry, trade, occupation, employment or business.
      \end{enumerate}

\item Right to equality:
\begin{enumerate}
    \item All citizens shall be equal before law. No person shall be denied the equal protection of law.
    \item No discrimination shall be made in the application of general laws on grounds of origin, religion, race, caste, tribe, sex, physical condition, condition of health, marital status, pregnancy, economic condition, language or region, ideology or on similar other grounds.
    \item The State shall not discriminate citizens on grounds of origin, religion, race, caste, tribe, sex, economic condition, language, region, ideology or on similar other grounds. Provided that nothing shall be deemed to prevent the making of special provisions by law for the protection, empowerment or development of the citizens including the socially or culturally backward women, Dalit, indigenous people, indigenous nationalities, Madhesi, Tharu, Muslim, oppressed class, Pichhadaclass, minorities, the marginalized, farmers, labours, youths, children, senior citizens, gender and sexual minorities, persons with disabilities, persons in pregnancy, incapacitated or helpless, backward region and indigent Khas Arya.
  Explanation: For the purposes of this Part and Part 4, "indigent" means a person who earns income less than that specified by the Federal law.
    \item No discrimination shall be made on the ground of gender with regard to remuneration and social security for the same work.
    \item All offspring shall have the equal right to the ancestral property without discrimination on the ground of gender.
\end{enumerate}

\item Right to communication:
\begin{enumerate}
    \item No publication and broadcasting or dissemination or printing of any news item, editorial, feature article or other reading, audio and audio-visual material through any means whatsoever including electronic publication, broadcasting and printing shall be censored. Provided that nothing shall be deemed to prevent the making of Acts to impose reasonable restrictions on any act which may undermine the sovereignty, territorial integrity, nationality of Nepal or the harmonious relations between the Federal Units or the harmonious relations between various castes, tribes, religions or communities, or on any act of sedition, defamation or contempt of court or incitement to an offence, or on any act which may be contrary to public decency or morality, on any act of hatred to labour and on any act of incitement to caste-based untouchability as well as gender discrimination.
    \item No radio, television, on-line or other form of digital or electronic equipment, press or other means of communication publishing, broadcasting or printing any news item, feature, editorial, article, information or other material shall be closed or seized nor shall registration thereof be cancelled nor shall such material be seized by the reason of publication, broadcasting or printing of such material through any audio, audio-visual or electronic equipment. Provided that nothing contained in this clause shall be deemed to prevent the making of an Act to regulate radio, television, online or any other form of digital or electronic equipment, press or other means of communication.
    \item No means of communication including the press, electronic broadcasting and telephone shall be interrupted except in accordance with law.
\end{enumerate}

\item Rights relating to justice:
\begin{enumerate}
    \item No person shall be detained in custody without informing him or her of the ground for his or her arrest.
    \item Any person who is arrested shall have the right to consult a legal practitioner of his or her choice from the time of such arrest and to be defended by such legal practitioner. Any consultation made by such person with, and advice given by, his or her legal practitioner shall be confidential. Provided this clause shall not apply to a citizen of an enemy state. Explanation: For the purpose of this clause, "legal practitioner" means any person who is authorized by law to represent any person in any court.
    \item Any person who is arrested shall be produced before the adjudicating authority within a period of twenty-four hours of such arrest, excluding the time necessary for the journey from the place of arrest to such authority; and any such person shall not be detained in custody except on the order of such authority. Provided that this clause shall not apply to a person held in preventive detention and to a citizen of an enemy state.
    \item No person shall be liable for punishment for an act which was not punishable by the law in force when the act was committed nor shall any person be subjected to a punishment greater than that prescribed by the law in force at the time of the commission of the offence.
    \item Every person charged with an offence shall be presumed innocent until proved guilty of the offence.
    \item No person shall be tried and punished for the same offence in a court more than once.
    \item No person charged with an offence shall be compelled to testify against himself or herself.
    \item Every person shall have the right to be informed of any proceedings taken against him or her.
    \item Every person shall have the right to a fair trial by an independent, impartial and competent court or judicial body.
    \item Any indigent party shall have the right to free legal aid in accordance with law.
\end{enumerate}

\item Right of victim of crime:
\begin{enumerate}
    \item A victim of crime shall have the right to get information about the investigation and proceedings of a case in which he or she is the victim.
    \item A victim of crime shall have the right to justice including social rehabilitation and compensation in accordance with law.
\end{enumerate}

\item Right against torture:
\begin{enumerate}
    \item No person who is arrested or detained shall be subjected to physical or mental torture or to cruel, inhuman or degrading treatment.
    \item Any act mentioned in clause (1) shall be punishable by law, and any person who is the victim of such treatment shall have the right to obtain compensation in accordance with law.
\end{enumerate}

\item Right against preventive detention:
\begin{enumerate}
    \item No person shall be held under preventive detention unless there is a sufficient ground of the existence of an immediate threat to the sovereignty, territorial integrity or public peace and order of Nepal.
    \item Information about the situation of a person who is held under preventive detention pursuant to clause (1) must be given immediately to his or her family members or relatives. Provided that this clause shall not apply to a citizen of an enemy state.
    \item If the authority making preventive detention holds any person under preventive detention contrary to law or in bad faith, the person held under preventive detention shall have the right to obtain compensation in accordance with law.
\end{enumerate}

\item Right against untouchability and discrimination:
\begin{enumerate}
    \item No person shall be subjected to any form of untouchability or discrimination in any private and public places on grounds of his or her origin, caste, tribe, community, profession, occupation or physical condition.
    \item In producing or distributing any goods, services or facilities, no person belonging to any particular caste or tribe shall be prevented from purchasing or acquiring such goods, services or facilities nor shall such goods, services or facilities be sold, distributed or provided only to the persons belonging to any particular caste or tribe.
    \item No act purporting to demonstrate any person or community as superior or inferior on grounds of origin, caste, tribe or physical condition or justifying social discrimination on grounds of caste, tribe or untouchability or propagating ideology based on untouchability and caste based superiority or hatred or encouraging caste-based discrimination in any manner whatsoever shall be allowed.
    \item No discrimination in any form shall be allowed at a workplace with or without making untouchability on the ground of caste.
    \item Any act of untouchability and discrimination in any for committed in contravention of this Article shall be punishable by law as a severe social offence, and the victim of such act shall have the right to obtain compensation in accordance with law.
\end{enumerate}

\item Right relating to property:
\begin{enumerate}
    \item Every citizen shall, subject to law, have the right to acquire, own, sell, dispose, acquire business profits from, and otherwise deal with, property. Provided that the State may levy tax on property of a person, and tax on income of a person in accordance with the concept of progressive taxation. Explanation: For the purposes of this Article, "property" means any form of property including movable and immovable property, and includes an intellectual property right.
    \item The State shall not, except for public interest, requisition, acquire, or otherwise create any encumbrance on, property of a person. Provided that this clause shall not apply to any property acquired by any person illicitly.
    \item The basis of compensation to be provided and procedures to be followed in the requisition by the State of property of any person for public interest in accordance with clause (2) shall be as provided for in the Act.
    \item The provisions of clauses (2) and (3) shall not prevent the State from making land reforms, management and regulation in accordance with lawfor the purposes of enhancement of product and productivity of lands, modernization and commercialization of agriculture, environment protection and planned housing and urban development.
    \item Nothing shall prevent the State from using the property of any person, which it has requisitioned for public interest in accordance with clause (3), for any other public interest instead of such public interest.
\end{enumerate}

\item Right to freedom of religion:
\begin{enumerate}
    \item Every person who has faith in religion shall have the freedom to profess, practice and protect his or her religion according to his or her conviction.
    \item Every religious denomination shall have the right to operate and protect its religious sites and religious Guthi (trusts).
    \item[] Provided that nothing shall be deemed to prevent the regulation, by making law, of the operation and protection of religious sites and religious trusts and management of trust properties and lands.
    \item No person shall, in the exercise of the right conferred by this Article, do, or cause to be done, any act which may be contrary to public health, decency and morality or breach public peace, or convert another person from one religion to another or any act or conduct that may jeopardize other's religion and such act shall be punishable by law.
\end{enumerate}

\item Right to information: Every citizen shall have the right to demand and receive information on any matter of his or her interest or of public interest. Provided that no one shall be compelled to provide information on any matter of which confidentiality must be maintained in accordance with law.
\item Right to privacy: The privacy of any person, his or her residence, property, document, data, correspondence and matters relating to his or her character shall, except in accordance with law, be inviolable.
\item Right against exploitation:
\begin{enumerate}
    \item Every person shall have the right against exploitation.
    \item No person shall be exploited in any manner on the grounds of religion, custom, tradition, usage, practice or on any other grounds.
    \item No one shall be subjected to trafficking nor shall one be held in slavery or servitude.
    \item No one shall be forced to work against his or her will. Provided that nothing shall be deemed to prevent the making of law empowering the State to require citizens to perform compulsory service for public purposes.
    \item Act contrary to clauses (3) and (4) shall be punishable by law and the victim shall have the right to obtain compensation from the perpetrator in accordance with law.
\end{enumerate}

\item Right to clean environment:
\begin{enumerate}
    \item Every citizen shall have the right to live in a clean and healthy environment.
    \item The victim shall have the right to obtain compensation, in accordance with law, for any injury caused from environmental pollution or degradation.
    \item This Article shall not be deemed to prevent the making of necessary legal provisions for a proper balance between the environment and development, in development works of the nation.
\end{enumerate}

\item Right relating to education:
\begin{enumerate}
    \item Every citizen shall have the right of access to basic education.
    \item Every citizen shall have the right to get compulsory and free education up to the basic level and free education up to the secondary level from the State.
    \item The citizens with disabilities and the economically indigent citizens shall have the right to get free higher education in accordance with law.
    \item The visually impaired citizens shall have the right to get free education through brail script and the citizens with hearing or speaking impairment, to get free education through sign language, in accordance with law.
    \item Every Nepalese community residing in Nepal shall have the right to get education in its mother tongue and, for that purpose, to open and operate schools and educational institutes, in accordance with law.
\end{enumerate}

\item Right to language and culture:
\begin{enumerate}
    \item Every person and community shall have the right to use their languages.
    \item Every person and community shall have the right to participate in the cultural life of their communities.
    \item Every Nepalese community residing in Nepal shall have the right to preserve and promote its language, script, culture, cultural civilization and heritage.
\end{enumerate}

\item Right to employment:
\begin{enumerate}
    \item Every citizen shall have the right to employment. The terms and conditions of employment, and unemployment benefit shall be as provided for in the Federal law.
    \item Every citizen shall have the right to choose employment.
\end{enumerate}

\item Right to labour:
\begin{enumerate}
    \item Every labourer shall have the right to practice appropriate labour. Explanation: For the purposes of this Article, "labourer" means a labourer or worker who does physical or mental work for an employer in consideration for remuneration.
    \item Every labourer shall have the right to appropriate remuneration, facilities and contributory social security.
    \item Every labourer shall have the right to form and join trade unions and to engage in collective bargaining, in accordance with law.
\end{enumerate}

\item Right relating to health:
\begin{enumerate}
    \item Every citizen shall have the right to free basic health services from the State, and no one shall be deprived of emergency health services.
    \item Every person shall have the right to get information about his or her medical treatment.
    \item Every citizen shall have equal access to health services.
    \item Every citizen shall have the right of access to clean drinking water and sanitation.
\end{enumerate}

\item Right relating to food:
\begin{enumerate}
    \item Every citizen shall have the right relating to food.
    \item Every citizen shall have the right to be safe from the state of being in danger of life from the scarcity of food.
    \item Every citizen shall have the right to food sovereignty in accordance with law.
\end{enumerate}

\item Right to housing:
\begin{enumerate}
    \item Every citizen shall have the right to an appropriate housing.
    \item No citizen shall be evicted from the residence owned by him or her nor shall his or her residence be infringed except in accordance with law.
\end{enumerate}

\item Rights of women:
\begin{enumerate}
    \item Every woman shall have equal lineage right without gender based discrimination.
    \item Every woman shall have the right to safe motherhood and reproductive health.
    \item No woman shall be subjected to physical, mental, sexual, psychological or other form of violence or exploitation on grounds of religion, social, cultural tradition, practice or on any other grounds. Such act shall be punishable by law, and the victim shall have the right to obtain compensation in accordance with law.
    \item Women shall have the right to participate in all bodies of the State on the basis of the principle of proportional inclusion.
    \item Women shall have the right to obtain special opportunity in education, health, employment and social security, on the basis of positive discrimination.
    \item The spouse shall have the equal right to property and family affairs.
\end{enumerate}

\item Rights of the child:
\begin{enumerate}
    \item Every child shall have the right to name and birth registration along with his or her identity.
    \item Every child shall have the right to education, health, maintenance, proper care, sports, entertainment and overall personality development from the families and the State.
    \item Every child shall have the right to elementary child development and child participation.
    \item No child shall be employed to work in any factory, mine or engaged in similar other hazardous work.
    \item No child shall be subjected to child marriage, transported illegally, abducted/kidnapped or taken in hostage.
    \item No child shall be recruited or used in army, police or any armed group, or be subjected, in the name of cultural or religious traditions, to abuse, exclusion or physical, mental, sexual or other form of exploitation or improper use by any means or in any manner.
    \item No child shall be subjected to physical, mental or any other form of torture in home, school or other place and situation whatsoever.
    \item Every child shall have the right to juvenile friendly justice.
    \item The child who is helpless, orphan, with disabilities, conflict victim, displaced or vulnerable shall have the right to special protection and facilities from the State.
    \item Any act contrary to in clauses (4), (5), (6) and (7) shall be punishable by law, and a child who is the victim of such act shall have the right to obtain compensation from the perpetrator, in accordance with law.
\end{enumerate}

\item Rights of Dalit:
\begin{enumerate}
    \item The Dalit shall have the right to participate in all bodies of the State on the basis of the principle of proportional inclusion. Special provision shall be made by law for the empowerment, representation and participation of the Dalit community in public services as well as other sectors of employment.
    \item Provision of free education with scholarship, from primary to higher education, shall be made by law for the Dalit students. Special provision shall be made by law for the Dalit in technical and vocational education.
    \item Special provision shall be made by law in order to provide health and social security to the Dalit community.
    \item The Dalit community shall have the right to use, protect and develop their traditional occupation, knowledge, skill and technology. The State shall accord priority to the Dalit community in modern business related with their traditional occupation and provide skills and resources required therefore.
    \item The State shall once provide land to the landless Dalit in accordance with law.
    \item The State shall, in accordance with law, arrange settlement for the Dalit who do not have housing.
    \item The facilities conferred by this Article to the Dalit community must be distributed in a just manner so that the Dalit women, men and Dalit in all communities can obtain such facilities proportionately.
\end{enumerate}

\item Rights of senior citizens: The senior citizens shall have the right to special protection and social security from the State.

\item Right to social justice:
\begin{enumerate}
    \item The socially backward women, Dalit, indigenous people, indigenous nationalities, Madhesi, Tharu, minorities, persons with disabilities, marginalized communities, Muslims, backward classes, gender and sexual minorities, youths, farmers, labourers, oppressed or citizens of backward regions and indigent Khas Aryashall have the right to participate in the State bodies on the basis of inclusive principle.
    \item The indigent citizens and citizens of the communities on the verge of extinction shall have the right to get special opportunities and benefits in education, health, housing, employment, food and social security for their protection, upliftment, empowerment and development.
    \item The citizens with disabilities shall have the right to live with dignity and honour, with the identity of their diversity, and have equal access to public services and facilities.
    \item Every farmer shall have the right to have access to lands for agro activities, select and protect local seeds and agro species which have been used and pursued traditionally, in accordance with law.
    \item The families of the martyrs who have sacrificed their life, persons who were forced to disappear, and those who became disabled and injured in all people's movements, armed conflicts and revolutions that have been carried out for progressive democratic changes in Nepal, democracy fighters, conflict victims and displaced ones, persons with disabilities, the injured and victims shall have the right to get a prioritized opportunity, with justice and due respect, in education, health, employment, housing and social security, in accordance with law.
\end{enumerate}

\item Right to social security: The indigent citizens, incapacitated and helpless citizens, helpless single women, citizens with disabilities, children, citizens who cannot take care themselves and citizens belonging to the tribes on the verge of extinction shall have the right to social security, in accordance with law.

\item Rights of the consumer:
\begin{enumerate}
    \item Every consumer shall have the right to obtain quality goods and services.
    \item A person who has suffered injury from any substandard goods or services shall have the right to obtain compensation in accordance with law.
\end{enumerate}

\item Right against exile: No citizen shall be exiled.
\item Right to constitutional remedies: There shall be a right to obtain constitutional remedies in the manner set forth in Article 133 or 144 for the enforcement of the rights conferred by this Part.
\item Implementation of fundamental rights: The State shall, as required, make legal provisions for the implementation of the rights conferred by this Part, within three years of the commencement of this Constitution.
\item Duties of citizens:
\begin{enumerate}
    \item[] Every citizen shall have the following duties:
    \item to safeguard the nationality, sovereignty and integrity of Nepal, while being loyal to the nation,
    \item to abide by the Constitution and law,
    \item to render compulsory service as and when the State so requires,
    \item to protect and preserve public property.
\end{enumerate}
\end{enumerate}
  \end{solution}

\end{questions}
